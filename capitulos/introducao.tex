\chapter{Introdução} \label{cap:introducao}

Com o aumento da quantidade de dispositivos móveis no mercado, a demanda por aplicações móveis também aumentou \cite{cevallos_case_2014}.
De início, a única opção que havia era desenvolver aplicações específicas para uma plataforma, utilizando todo o ambiente daquela plataforma, por exemplo,
no caso do iOS, a linguagem de programação era o \textit{Objective-C} com a \textit{IDE (Integrated Development Environment) Xcode} e o \textit{SDK (Software Development Kit)} do iOS \cite{heitkotter_comparing_2013}.

Segundo \citeonline{prezotto_estudo_2014}, essa forma de desenvolvimento é conhecida como desenvolvimento \textbf{nativo} e é aquele no qual um aplicativo é projetado e construído especificamente para uma plataforma. 
Todas as funcionalidades da plataforma estão disponíveis sem restrição e existem padrões de interface gráfica e experiência de usuário específicos, que ajudam o usuário a 
entender como aquele aplicativo funciona, já que todos os outros aplicativos daquela plataforma seguem os mesmos padrões \cite{corral_ant_2012}. 

No entanto, com o aumento exponencial da demanda por aplicativos, surgiu a necessidade de desenvolvimento rápido para muitas plataformas distintas, o que era caro e difícil,
visto que, cada plataforma tinha um ambiente de desenvolvimento completamente diferente das demais, o que demandava da equipe muitos conhecimentos diferentes
para poder desenvolver e manter vários aplicativos \cite{prezotto_estudo_2014}.

Segundo \citeonline{heitkotter_comparing_2013}, com esse novo panorama do mercado, surgiu a necessidade da criação de apenas um \textit{app} que pudesse ser executado 
em várias plataformas. Essa forma de desenvolvimento ficou conhecida como \textbf{multiplataforma} (\textit{cross-platform}). 
Consiste na criação de uma página \textit{web}, normalmente utilizando \textit{HTML (HiperText Markup Language)}, \textit{CSS (Cascade Style Sheet)} e \textit{JavaScript}, que pode ser mostrada dentro de uma \textit{web view} 
embutida em um aplicativo nativo, ou seja, é feito apenas um código, que é executado e mostrado em um \textit{container} dentro de um aplicativo nativo \cite{stark_building_2010, heitkotter_comparing_2013}. 

Dessa forma, é possível desenvolver um mesmo aplicativo que será executado dentro de várias plataformas diferentes, sem a necessidade de ter uma equipe especialista em cada plataforma.
No entanto, não é possível ter acesso total às funcionalidades da plataforma, assim como no desenvolvimento nativo, e também não existe padrão de interface ou de experiência de usuário que sirva
para várias plataformas ao mesmo tempo, podendo causar uma sensação ruim na usabilidade e experiência de uso do aplicativo \cite{corral_ant_2012}.

O presente trabalho visa auxiliar desenvolvedores, no momento inicial da criação de aplicativos 
para dispositivos móveis, a escolherem a melhor abordagem que se encaixa no contexto em que estão inseridos,
considerando as reais vantagens e desvantagens presentes atualmente no desenvolvimento multiplataforma.

\section{Justificativa}\label{sec:justificativa}

Sabendo que existem diversas plataformas diferentes para dispositivos móveis e que há uma crescente 
necessidade de mercado de abarcar o maior número de plataformas, o mais rápido, com a
melhor qualidade e menor custo possíveis, o presente trabalho visa auxiliar desenvolvedores \textit{mobile} 
a compreender melhor o mundo do desenvolvimento móvel e tomar a decisão de 
desenvolver nativamente ou multiplataforma com mais consciência das vantagens e desvantagens de cada abordagem.

\section{Objetivos} \label{sec:objetivos}

O objetivo deste trabalho é definir, por meio de pesquisa e comprovação empírica, as vantagens e desvantagens de desenvolvimento multiplataforma de aplicações móveis em relação ao desenvolvimento nativo. 
Além desse objetivo principal, existem ainda outros objetivos secundários que precisam ser atingidos. São eles: 
\begin{itemize}
    \item Pesquisar sobre o desenvolvimento de aplicativos móveis;
    \item Compreender a arquitetura das duas principais plataformas móveis da atualidade (iOS e Android);
    \item Desenvolver uma réplica de um aplicativo criado nativamente para uma plataforma, utilizando tecnologias multiplataforma;
    \item Comparar o desenvolvimento do aplicativo multiplataforma com o aplicativo nativo;
    \item Comparar funcionalidades chave desenvolvidas nas duas abordagens;
    \item Confrontar resultados obtidos empiricamente com opiniões de especialistas da área;
\end{itemize}
Com base nos objetivos traçados para o presente trabalho, a questão problema a ser respondida é:
\begin{center}
    \textit{Quais as vantagens e desvantagens do desenvolvimento multiplataforma de aplicações móveis em relação ao desenvolvimento nativo?}
\end{center}

\section{Organização do Trabalho}\label{sec:organizacao}

Este trabalho foi dividido em seis capítulos. %, sendo o capítulo~\ref{cap:introducao} de introdução. 
No Capítulo~\ref{cap:referencialteorico}, é feito um estudo sobre como é feito hoje e como evoluiu ao longo dos últimos anos, o desenvolvimento de aplicações 
móveis. Também foi feito um comparativo para avaliar vantagens e desvantagens entre as duas abordagens de desenvolvimento trazidas pela literatura. 
No Capítulo~\ref{cap:metodologia}, é apresentada a metodologia adotada para o desenvolvimento do trabalho, o planejamento feito e como será executado, para que 
o trabalho possa ser replicado e/ou continuado futuramente.  
No Capítulo~\ref{cap:estudodecaso}, é feita uma análise de exemplo de uso, realizado durante o trabalho, sobre o desenvolvimento de aplicativos móveis.
No Capítulo~\ref{cap:analise_exploratoria}, é feita a comparação do desenvolvimento de algumas funcionalidades comuns em muitos aplicativos conhecidos, e os dados confrontados com as opiniões dos especialistas a respeito 
dos dados obtidos. Por fim, no Capítulo~\ref{cap:consideracoespreliminares}, apresentam-se a resposta da questão problema, as conclusões do trabalho e uma lista de possíveis limitações e trabalhos futuros.  
% considerações preliminares acerca do desenvolvimento multiplataforma e da realização da análise do exemplo de uso, bem como apresenta-se um cronograma preliminar do desenvolvimento da continuação deste trabalho no Trabalho de Conclusão de Curso II. 