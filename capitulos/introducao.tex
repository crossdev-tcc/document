\chapter{Introdução} \label{cap:introducao}

Com o aumento da quantidade de dispositivos móveis no mercado, a demanda por aplicações móveis também aumentou.
De início, a única opção que havia era desenvolver aplicações específicas para uma plataforma, utilizando todo o ambiente daquela plataforma, por exemplo,
no caso do \textit{iOS}, a liguagem de programação era o \textit{Objective-C} com a \textit{IDE Xcode} e o \textit{SDK} do \textit{iOS}.


Essa forma de desenvolvimento é conhecida como desenvolvimento \textbf{nativo} e é aquele no qual um aplicativo é projetado e construído especificamente para uma plataforma. 
Todas as funcionalidades da plataforma estão disponíveis sem restrição e existem padrões de interface gráfica e experiência de usuário específicos, que ajudam o usuário a 
entender como aquele aplicativo funciona, já que todos os outros aplicativos daquela plataforma seguem os mesmos padrões. 


No entanto, com o aumento exponencial da demanda por aplicativos, surgiu a necessidade de desenvolvimento rápido para muitas plataformas distintas, o que era caro e difícil,
visto que, cada plataforma tinha um ambiente de desenvolvimento completamente diferente das demais, o que fazia com que a equipe tivesse que ter muitos conhecimentos diferentes
para poder desenvolver e manter vários aplicativos.  


Com esse novo panorama do mercado, surgiu a necessidade da criação de apenas um \textit{app} que pudesse ser executado em várias plataformas. Essa forma de desenvolvimento ficou conhecida como 
\textbf{multiplataformas} (\textit{cross-platform}) e consiste na criação de uma página \textit{web}, 
normalmente utilizando \textit{HTML} e \textit{CSS}. Essa página pode ser mostrada dentro de uma \textit{web view}%vista diretamente pelo \textit{browser} que o dispositivo possui ou ainda ser 
embutida em um aplicativo nativo, ou seja, é feito apenas um código, que é executado e mostrado em um \textit{container} dentro de um aplicativo nativo. 


Dessa forma, é possível desenvolver um mesmo aplicativo que será executado dentro de várias plataformas diferentes, sem a necessidade de ter uma equipe especialista em cada plataforma, pois a única 
parte nativa do aplicativo, é uma \textit{web view} simples, o que não requer grandes conhecimentos específicos em cada plataforma.


No entanto, não é possível ter acesso total às funcionalidades da plataforma, assim como no desenvolvimento nativo, e também não existe padrão de interface ou de experiência de usuário que sirva
para várias plataformas ao mesmo tempo, podendo causar uma sensação ruim na usabilidade e experiência de uso do aplicativo.

\textcolor{red}{...MELHORAR E CONTINUAR A INTRODUÇAO...}

\section{Justificativa}\label{sec:justificativa}

A justificativa

\section{Objetivos} \label{sec:objetivos}

São apresentados, nesta seção, os objetivos gerais e específicos, deste trabalho. 
\textcolor{red}{\textbf{\textit{REVER ESSES OBJETIVOS, TALVEZ TENTAR FAZER PARECIDO OU IGUAL AO DO KANASHIRO!!!}}}

\subsection{Objetivos Gerais}  \label{subsec:objetivos_gerais}

Definir por meio de pesquisa e comprovação empírica as vantagens e desvantagens de desenvolvimento multiplataformas de aplicações móveis em relação ao desenvolvimento nativo. 

\subsection{Objetivos Específicos}  \label{subsec:objetivos_especificos}

Os objetivos específicos deste trabalho são listados a seguir.

\begin{itemize}
    \item Pesquisar sobre o desenvolvimento de aplicativos móveis;
    \item Pesquisar sobre Linha de Produtos de \textit{Software};
    \item Compreender a arquitetura das duas principais plataformas móveis da atualidade (\textit{iOS} e \textit{Android});
    \item Desenvolver uma réplica de um aplicativo criado nativamente, para a plataforma \textit{iOS}, utilizando tecnologias multiplataformas, para \textit{iOS} e \textit{Android};
    \item Comparar o desenvolvimento do aplicativo multiplataformas com o aplicativo nativo;
\end{itemize}

\section{Organização do Trabalho}\label{sec:organizacao}

Este trabalho foi dividido em nove capítulos, sendo o capítulo~\ref{cap:introducao} de introdução. 
Na Seção~\ref{cap:referencialteorico}, é feito um estudo sobre como é e como evoluiu o desenvolvimento de aplicações móveis ao longo dos últimos anos. 
Na Seção~\ref{cap:desenvolvimentonativo}, é feito um estudo sobre desenvolvimento nativo de aplicações móveis.
Na Seção~\ref{cap:desenvolvimentomulti}, é feito um estudo sobre desenvolvimento multiplataformas de aplicações móveis.
Na Seção~\ref{cap:comparativo}, é feito um comparativo para avaliar vantagens, desvantagens, semelhanças e diferenças entre o desenvolvimento de aplicativos móveis de maneira nativa e multiplataformas.  
Na Seção~\ref{cap:linhaproduto}, é feito um estudo sobre \textit{LPS} para avaliar se o densenvolvimento multiplataformas pode ser considerado como um desenvolvimento de linha de produtos de \textit{software}.
Na Seção~\ref{cap:metodologia}, é apresentada a metodologia adotada para o desenvolvimento do trabalho.  
Na Seção~\ref{cap:estudodecaso}, é descrito o estudo de caso, realizado durante o trabalho, sobre o desenvolvimento de aplicativos móveis.
Por fim, na Seção~\ref{cap:consideracoespreliminares}, apresentam-se as considerações preliminares acerca do desenvolvimento multiplataformas e da realização do estudo de caso, bem como apresenta-se um 
cronograma preliminar do desenvolvimento da continuação desse trabalho no Trabalho de Conclusão de Curso II. 