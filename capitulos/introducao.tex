\chapter{Introdução} \label{cap:introducao}

Com o aumento da quantidade de dispositivos móveis no mercado, a demanda por aplicações móveis também aumentou.
De início, a única opção que havia era desenvolver aplicações específicas para uma plataforma, utilizando todo o ambiente daquela plataforma, por exemplo,
no caso do \textit{iOS}, a liguagem de programação era o \textit{Objective-C} com a \textit{IDE Xcode} e o \textit{SDK} do \textit{iOS}.

Essa forma de desenvolvimento é conhecida como desenvolvimento \textbf{nativo} e é aquele no qual um aplicativo é projetado e construído especificamente para uma plataforma. 
Todas as funcionalidades da plataforma estão disponíveis sem restrição e existem padrões de interface gráfica e experiência de usuário específicos, que ajudam o usuário a 
entender como aquele aplicativo funciona, já que todos os outros aplicativos daquela plataforma seguem os mesmos padrões. 

No entanto, com o aumento exponencial da demanda por aplicativos, surgiu a necessidade de desenvolvimento rápido para muitas plataformas distintas, o que era caro e difícil,
visto que, cada plataforma tinha um ambiente de desenvolvimento completamente diferente das demais, o que demandava da equipe muitos conhecimentos diferentes
para poder desenvolver e manter vários aplicativos.  

Com esse novo panorama do mercado, surgiu a necessidade da criação de apenas um \textit{app} que pudesse ser executado em várias plataformas. Essa forma de desenvolvimento ficou conhecida como 
\textbf{multiplataformas} (\textit{cross-platform}) e consiste na criação de uma página \textit{web}, 
normalmente utilizando \textit{HTML} e \textit{CSS}. Essa página pode ser mostrada dentro de uma \textit{web view} %vista diretamente pelo \textit{browser} que o dispositivo possui ou ainda ser 
embutida em um aplicativo nativo, ou seja, é feito apenas um código, que é executado e mostrado em um \textit{container} dentro de um aplicativo nativo. 

Dessa forma, é possível desenvolver um mesmo aplicativo que será executado dentro de várias plataformas diferentes, sem a necessidade de ter uma equipe especialista em cada plataforma, pois a única 
parte nativa do aplicativo, é uma \textit{web view} simples, o que não requer grandes conhecimentos específicos em cada plataforma.

No entanto, não é possível ter acesso total às funcionalidades da plataforma, assim como no desenvolvimento nativo, e também não existe padrão de interface ou de experiência de usuário que sirva
para várias plataformas ao mesmo tempo, podendo causar uma sensação ruim na usabilidade e experiência de uso do aplicativo.

\textcolor{red}{...MELHORAR E CONTINUAR A INTRODUÇAO...}

\section{Justificativa}\label{sec:justificativa}

Sabendo que existem muitas plataformas diferentes para dispositivos móveis atualmente, e há uma crescente necessidade de mercado de abarcar o maior número possível de plataformas, o mais rápido possível, com a
melhor qualidade e menor custo possíveis, o presente trabalho visa auxiliar desenvolvedores \textit{mobile} a compreender melhor o mundo do desenvolvimento móvel e tomar a decisão de 
desenvolver nativamente ou multiplataformas com mais consciência das vantagens e desvantagens de cada abordagem.

\section{Objetivos} \label{sec:objetivos}

O objetivo deste trabalho é definir por meio de pesquisa e comprovação empírica as vantagens e desvantagens de desenvolvimento multiplataformas de aplicações móveis em relação ao desenvolvimento nativo. 
Com base nesse objetivo principal, existem ainda outros objetivos secundários que precisam ser atingidos para 
\begin{itemize}
    \item Pesquisar sobre o desenvolvimento de aplicativos móveis;
    \item Pesquisar sobre Linha de Produtos de \textit{Software};
    \item Compreender a arquitetura das duas principais plataformas móveis da atualidade (\textit{iOS} e \textit{Android});
    \item Desenvolver uma réplica de um aplicativo criado nativamente, para a plataforma \textit{iOS}, utilizando tecnologias multiplataformas, para \textit{iOS} e \textit{Android};
    \item Comparar o desenvolvimento do aplicativo multiplataformas com o aplicativo nativo;
\end{itemize}
Com base nos objetivos traçados para o presente trabalho, a questão problema a ser respondida é:
\begin{center}
    \textit{Quais as vantagens e desvantagens do desenvolvimento multiplataforma de aplicações móveis em relação ao desenvolvimento nativo?}
\end{center}
A partir da pergunta feita, foram formuladas duas hipóteses acerca do tema, que são apresentadas a seguir.
\begin{itemize}
    \item \textbf{H1}: As plataformas \textit{cross-development} estão evoluídas ao ponto de apresentarem mais vantagens do que desvantagens.
    \item \textbf{H2}: Ao longo do tempo, as possíveis desvantagens do desenvolvimento multiplataformas serão sanadas ou mitigadas.
\end{itemize}

\section{Organização do Trabalho}\label{sec:organizacao}

Este trabalho foi dividido em cinco capítulos. %, sendo o capítulo~\ref{cap:introducao} de introdução. 
No Capítulo~\ref{cap:referencialteorico}, é feito um estudo sobre como é e como evoluiu o desenvolvimento de aplicações 
móveis ao longo dos últimos anos, assim como um comparativo para avaliar vantagens e desvantagens entre o desenvolvimento 
de aplicativos móveis de maneira nativa e multiplataformas. 
No Capítulo~\ref{cap:metodologia}, é apresentada a metodologia adotada para o desenvolvimento do trabalho.  
No Capítulo~\ref{cap:estudodecaso}, é descrito o estudo de caso, realizado durante o trabalho, sobre o desenvolvimento de aplicativos móveis.
Por fim, no Capítulo~\ref{cap:consideracoespreliminares}, apresentam-se as considerações preliminares acerca do desenvolvimento multiplataformas e da realização do estudo de caso, bem como apresenta-se um 
cronograma preliminar do desenvolvimento da continuação desse trabalho no Trabalho de Conclusão de Curso II. 