\chapter{Considerações Preliminares} \label{cap:consideracoespreliminares}

Ao longo da execução do trabalho, foi possível perceber que as ferramentas para desenvolvimento multiplataforma 
evoluíram muito desde sua criação, o que as tornaram, hoje, uma opção que deve ser considerada no momento da criação de um novo \textit{app}.

Ambas as abordagens de desenvolvimento móvel possuem suas vantagens e desvantagens, conforme apresentado nos capítulos anteriores deste trabalho. 
No entanto, antigamente as ferramentas multiplataforma apresentavam um \textit{gap} muito grande quando comparadas às ferramentas e ambientes nativos e com isso, 
dificilmente eram consideradas no momento do desenvolvimento de um aplicativo.

Todas as funcionalidades do aplicativo Mini Farma, que foram planejadas para serem desenvolvidas no ambiente multiplataforma, puderam ser desenvolvidas, não havendo quaisquer limitações 
quanto ao uso dos recursos nativos do dispositivo necessários para o projeto selecionado. O \textit{app} multiplataforma se assemelhou muito 
ao nativo em relação a aparência e usabilidade o que confirma a ideia de que as ferramentas multiplataforma estão cada vez mais se aproximando
das nativas apresentando, com o passar do tempo, mais vantagens do que desvantagens, mostrando ainda, que os gargalos antes vistos para essa forma de desenvolvimento, não condizem
mais com a realidade.

\begin{figure}[H]
	\centering
	\includegraphics[width=0.55\textwidth]{analiseapp}
	\caption[Fatores a serem avaliados no desenvolvimento móvel]{ Fatores a serem avaliados no desenvolvimento móvel.}
	\label{fig:analiseapp}
\end{figure}

Com o término da primeira parte deste trabalho, pôde-se concluir que o desenvolvimento móvel requer uma análise aprofundada de uma série de fatores, 
como mercado, público e tecnologias para decidir qual abordagem escolher.
É importante ressaltar que a abordagem nativa não é melhor que a multiplataforma ou vice-versa, sendo apenas distinta e deve-se avaliar qual utilizar caso a 
caso. Para cada situação existem fatores que devem ser avaliados de uma maneira conjunta e alguns desses fatores são listados e explicados a seguir e apresentados na Figura~\ref{fig:analiseapp}.

\begin{itemize}
    \item \textbf{Tipo e complexidade da aplicação}: cada aplicação possui requisitos diferentes e próprios que originam necessidades e dificuldades inerentes daquele aplicativo. Com isso, deve-se avaliar,
    com base nos requisitos da aplicação, qual abordagem suporta melhor o \textit{app};
    \item \textbf{\textit{Expertise} da equipe nas plataformas e seus ambientes}: cada equipe possui um conjunto único de habilidades e conhecimentos. No momento da escolha de uma abordagem, esses conhecimentos
    devem ser levados em consideração, visto que é a equipe de desenvolvimento que irá conceber o produto final. Se a equipe possui mais conhecimentos em uma abordagem do que em outra, isso pode ser 
    um indicativo de qual abordagem escolher;
    \item \textbf{Nicho de mercado que se quer atacar}: Cada plataforma móvel (iOS, Android, \textit{Windows Phone}, etc), domina uma parcela do mercado e possui um grupo de usuários com 
    características, opiniões, necessidades e gostos próprios inerentes à plataforma que usam. Com isso, no momento de criar um \textit{app} deve-se pensar para quem é esse aplicativo. Se ele for concebido 
    para suprir uma demanda de um grupo específico, talvez não haja a necessidade de criá-lo para várias plataformas;
    \item \textbf{Prazo de desenvolvimento}: Quanto mais plataformas para atender, maior é o tempo necessário para desenvolver a solução. Se o prazo do projeto for apertado para desenvolvimento de mais de uma solução 
    nativa, há de considerar o desenvolvimento multiplataforma, visto que apenas será codificada uma solução que poderá atender várias plataformas diferentes;
    \item \textbf{Capital disponível para investimento}: Desenvolver para plataformas nativas exige ambiente, infraestrutura e conhecimentos diferentes para cada plataforma. Dessa forma, quanto mais plataformas se 
    quer abarcar, mais custoso o projeto será. Uma solução multiplataforma pode ser mais viável economicamente dependendo da situação;
\end{itemize}

\section{Planos Futuros} \label{section:planosfuturos}

Na continuação deste trabalho, serão realizadas novas análises de exemplos de uso a fim de obter mais detalhes sobre as vantagens e desvantagens das abordagens de desenvolvimento móvel.
A seguir, é apresentado, na Tabela~\ref{tab:cronograma}, um cronograma preliminar das atividades a serem realizadas.

\begin{table}[h]
\resizebox{\textwidth}{!}{
\begin{tabular}{|l|c|c|c|c|c|c|}
\hline
\multicolumn{1}{|c|}{
\textbf{Atividade}}	                                            & \textbf{Julho} & \textbf{Agosto} & \textbf{Setembro} & \textbf{Outubro} & \textbf{Novembro} & \textbf{Dezembro}   \\ \hline
Recriar um aplicativo nativo Android em Ionic                   & X              & X               &                   &                &                  &                    \\ \hline
Propor um fluxograma de tomada de decisão                       &               & X               & X                  &                &                  &                    \\ \hline
Investigar relação com linha de produto de \textit{software}    &               & X               & X                  &                &                  &                    \\ \hline
Avaliar fluxograma proposto em um projeto Ionic                 &               &                & X                  & X               & X                 & X                   \\ \hline
Refinar análise de vantagens e desvantagens e do fluxograma	    &               &                &                   &                & X                 & X                   \\ \hline
\end{tabular}
}
\caption{Cronograma inicial para o TCC 2}
\label{tab:cronograma}
\end{table}

Será recriado um aplicativo, originalmente feito para a plataforma Android, utilizando o \textit{framework} Ionic a fim de aprimorar o comparativo entre o desenvolvimento nativo e multiplataforma e colher mais insumos
para a proposta de um fluxograma de tomada de decisão de qual abordagem é mais adequada para um dado contexto de desenvolvimento.

Uma vez feito o novo aplicativo, será proposto um fluxograma para auxiliar desenvolvedores a escolher de forma mais assertiva qual abordagem utilizar no contexto de desenvolvimento que estiver imerso.

Realizar pesquisas na área de linha de produto de \textit{software}, para investigar se há alguma relação com o desenvolvimento de aplicativos móveis.

A fim de validar o fluxograma proposto, o mesmo será utilizado em um projeto de evolução de um aplicativo Ionic para avaliar se o modelo de tomada de decisão está correto ou precisa de melhorias.

Ao final serão refinados a análise de vantagens e desvantagens das abordagens de desenvolvimento móvel e o fluxograma criado.