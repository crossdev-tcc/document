\chapter{Considerações Preliminares} \label{cap:consideracoespreliminares}

Ao longo da execução do trabalho, foi possível perceber que as ferramentas para desenvolvimento multiplataformas 
evoluíram muito desde sua criação, o que as tornaram, hoje, uma opção que deve ser considerada no momento da criação de um novo \textit{app}.

Ambas as abordagens de desenvolvimento móvel possuem suas vantagens e desvantagens, conforme apresentado nos capítulos anteriores desse trabalho. 
No entanto, antigamente as ferramentas \textit{cross} apresentavam um \textit{gap} muito grande quando comparadas às ferramentas e ambientes nativos e com isso, 
dificilmente eram consideradas no momento do desenvolvimento de um aplicativo.

Todas as funcionalidades do aplicativo Mini Farma, que foram planejadas para serem desenvolvidas no ambiente \textit{cross-plataform}, puderam ser desenvolvidas, não havendo quaisquer limitações 
quanto ao uso dos recursos nativos do dispositivo necessários para o projeto selecionado. O \textit{app} multiplataformas se assemelhou muito 
ao nativo em relação a aparência e usabilidade o que confirma a ideia de que as ferramentas multiplataformas estão cada vez mais se aproximando
das nativas apresentando, com o passar do tempo, mais vantagens do que desvantagens, mostrando ainda, que os gargalos antes vistos para essa forma de desenvolvimento, não condizem
mais com a realidade.

Com o término da primeira parte deste trabalho, pôde-se concluir que o desenvolvimento móvel requer uma análise aprofundada de uma série de fatores, 
como mercado, público e tecnologias para decidir qual abordagem escolher.
É importante ressaltar, que a abordagem nativa não é melhor que a \textit{cross-platform} ou vice-versa, sendo apenas distinta e deve-se avaliar qual utilizar caso a 
caso. Para cada situação existem fatores que devem ser avaliados de uma maneira conjunta e alguns desses fatores são listados e explicados a seguir.

\begin{itemize}
    \item \textbf{Tipo e complexidade da aplicação}: cada aplicação possui requisitos diferentes e próprios que originam necessidades e dificuldades inerentes daquele aplicativo. Com isso, deve-se avaliar,
    com base nos requisitos da aplicação, qual abordagem suporta melhor o \textit{app};
    \item \textbf{\textit{Expertise} da equipe nas plataformas e seus ambientes}: cada equipe possui um conjunto único de habilidades e conhecimentos. No momento da escolha de uma abordagem, esses conhecimentos
    devem ser levados em consideração, visto que é a equipe de desenvolvimento que irá conceber o produto final. Se a equipe, possui mais conhecimentos em uma abordagem do que em outra, isso pode ser 
    um indicativo de qual abordagem escolher;
    \item \textbf{Nicho de mercado que se quer atacar}: Cada plataforma móvel (iOS, Android, \textit{Windows Phone}, etc), domina uma parcela do mercado e possui um grupo de usuários com 
    características, opiniões, necessidades e gostos próprios inerentes à plataforma que usam. Com isso, no momento de criar um \textit{app} deve-se pensar para quem é esse aplicativo. Se ele for concebido 
    para suprir uma demanda de um grupo específico, talvez não haja a necessidade de criá-lo para várias plataformas;
    \item \textbf{Prazo de desenvolvimento}: Quanto mais plataformas para atender, maior é o tempo necessário para desenvolver a solução. Se o prazo do projeto for apertado para desenvolvimento de mais de uma solução 
    nativa, há de considerar o desenvolvimento multiplataformas, visto que apenas será codificada uma solução que poderá atender várias plataformas diferentes;
    \item \textbf{Capital disponível para investimento}: Desenvolver para plataformas nativas exige ambiente, infraestrutura e conhecimentos diferentes para cada plataforma. Dessa forma, quanto mais plataformas se 
    quer abarcar, mais custoso o projeto será. Uma solução multiplataformas pode ser mais viável economicamente dependendo da situação;
\end{itemize}

A Figura~\ref{fig:analiseapp} a seguir, resume os fatores que devem ser avaliados no momento da escolha da abordagem de desenvolvimento.
\begin{figure}[H]
  \centering
    \includegraphics[width=0.5\textwidth]{analiseapp}
    \caption[Fatores a serem avaliados no desenvolvimento móvel]{ Fatores a serem avaliados no desenvolvimento móvel. Fonte: Elaborado pelos autores.}
	\label{fig:analiseapp}
\end{figure}

\section{Planos Futuros} \label{section:planosfuturos}

Na continuação deste trabalho, o principal objetivo será responder a questão problema definida, comprovando ou refutando as hipóteses criadas.
Para atingir esse objetivo, serão comparados códigos de aplicativos, desenvolvidos nativamente ou multiplataforma, 
encontrados em repositórios públicos como Github, por exemplo. 
Em posse dos códigos, será possível analisar métricas de qualidade de código e juntamente com o embasamento teórico e empírico desse trabalho, 
criaremos um modelo de escolha para facilitar a tarefa dos desenvolvedores de escolherem qual a melhor abordagem deve ser utilizada.

A seguir, é apresentado um cronograma preliminar das atividades que serão realizadas na continuação deste trabalho.

\begin{table}[h]
\resizebox{\textwidth}{!}{
\begin{tabular}{|l|c|c|}
\hline
\multicolumn{1}{|c|}{
\textbf{Atividade}}	                & \textbf{Data de Início} & \textbf{Data de Término} \\ \hline
Reunir aplicativos em repositórios  & 01/08/2016              & 01/09/2016                    \\ \hline
Coletar dados de métricas           & 02/09/2016              & 01/10/2016                    \\ \hline
Analisar dados obtidos	            & 02/10/2016              & 01/11/2016                    \\ \hline
Criar modelo de escolha	            & 02/11/2016              & 01/12/2016                    \\ \hline 
Concluir documento	                & 01/12/2016              & 15/12/2016                    \\ \hline
\end{tabular}
}
\caption{Cronograma inicial para o TCC 2}
\label{tab:cronograma}
\end{table}

