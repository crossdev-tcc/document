\chapter{Metodologia} \label{metodologia}

\textcolor{red}{\textbf{\textit{---NÃO SEI SE ISSO VAI FICAR AQUI---}}}

O projeto do aplicativo Mini Farma foi executado contando com uma equipe de dois integrantes com 
conhecimento intermediário na plataforma \textit{iOS}. Os detalhes técnicos são listados a seguir.
\begin{itemize}
    \item \textbf{Máquinas}: Dois MacBook Pro Retina 13", processador \textit{Intel Core i5}, 8 GB de \textit{RAM};
    \item \textbf{Sistema Operacional}: \textit{Mac OS X Yosemite};
    \item \textbf{IDE}: \textit{Xcode} v6.4;
    \item \textbf{Linguagem de Programação}: \textit{Swift} v1.3;
    \item \textbf{Tempo de desenvolvimento}: Ambos os integrantes trabalharam no projeto quatro horas por dia de segunda à sexta por duas semanas;
\end{itemize}

A partir do aplicativo feito em \textit{iOS}, foi construída uma versão utilizando o \textit{Ionic} para as plataformas \textit{Android} e \textit{iOS}.
Para a criação dessa versão do \textit{app}, a equipe foi alterada, mas permaneceu com dois integrantes, que são os autores deste trabalho.
No entanto, os integrantes não possuiam qualquer conhecimento prévio no \textit{framework Ionic}
e apenas possuiam um conhecimento básico em \textit{HTML}, \textit{CSS} e \textit{JavaScript}. Os detalhes técnicos são listados a seguir.
   
\begin{itemize}
    \item \textbf{Máquinas}: Dois MacBook Pro Retina 13", processador \textit{Intel Core i5}, 8 GB de \textit{RAM};
    \item \textbf{Sistema Operacional}: \textit{Mac OS X El Capitan};
    \item \textbf{IDE}: \textit{WebStorm} v2016.1.1;
    \item \textbf{Linguagem de Programação}: \textit{JavaScript}. 
    \begin{itemize}
        \item \textit{HTML} e \textit{CSS} foram usados paralelamente para a criação da interface gráfica do aplicativo;
    \end{itemize}
    \item \textbf{Ambiente de suporte}: \textit{Google Chrome}, para \textit{debug} do aplicativo;
    \item \textbf{Tempo de desenvolvimento}: Ambos os integrantes trabalharam x horas;
\end{itemize}


\begin{comment}
Texto introdutório da seção -> Precisa?
ok - projeto feito com tais condicoes de equipe, tempo e equipamento
ok - agora vai ser feito com tais nova condicoes de equipe, tempo e equipamento
ok - pouco conhecimento em html, css e javascript
ok - pq o uso do mini farma, pq o código eh nosso, app pequeno, mas que usa varios recursos nativos do celular
ok - pq o uso do ionic e nao de outra, criterios de escolha
será que aqui fala das versoes do ios, ionic, cordova, angularjs? sim e explicar pq nao escolheu a versao 2 do ionic, pq eh recente e estavel
\end{comment}

\section{Planejamento da pesquisa} \label{sec:planejamento}

O planejamento

\subsection{Hipóteses} \label{subsec:hipoteses}

As hipóteses

\subsection{Seleção das plataformas} \label{subsec:selecaodasplataformas}

A escolha do \textit{Ionic framework} se deu por alguns fatores, listados a seguir. 
\begin{enumerate}
    \item Novo, open source, gratuito,
    \item Ambiente de apoio gratuito e robusto
    \item Utiliza AngularJS e Cordova que são dois frameworks estáveis, muito conhecidos e utilizados
    \item Muitos plugins disponiveis e comunidade muito ativa para resolução de problemas %http://codepen.io/b00stup/post/why-ionic-framework
    \item Framework que consegue imitar melhor a aparencia e usabilidade dos dispositivos nativos %http://www.joshmorony.com/8-reasons-why-im-glad-i-switched-to-the-ionic-framework/
\end{enumerate}
 
\begin{comment}
Porque Ionic? Porque não, outro? Quais os criterios usados para escolher o Ionic?
Ionic é open source, gratuito, muito focado em performance e possui html e css otimizados para dispositivos móveis.
\end{comment}

\subsection{Seleção do projeto} \label{subsec:selecaodoprojeto}

A escolha do aplicativo Mini Farma se deu por alguns fatores, listados a seguir:

\begin{enumerate}
    \item É de propriedade de um dos autores deste trabalho;
    \item Explora algumas capacidades do dispositivo, que serão explicitadas mais adiante;
    \item Aplicativo pequeno e simples;
\end{enumerate}
 