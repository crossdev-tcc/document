\chapter{Metodologia} \label{cap:metodologia}
%Nesta seção é apresentada a metodologia utilizada para o desenvolvimento desse trabalho.
Para a realização deste trabalho foi feita, inicialmente, uma pesquisa sobre desenvolvimento de aplicações móveis. 
Para isso, foram utilizados artigos e materiais \textit{on-line} para definição de como é esse desenvolvimento
de maneira nativa e multiplataformas e, com isso, foi feita uma comparação entre os dois modos. 


Após o estudo e a definição de como se dá esse desenvolvimento específico de \textit{software},
foi feita uma pesquisa de plataformas para desenvolvimento \textit{cross-platform} e posteriormente uma ferramenta foi escolhida, no caso o \textit{Ionic}. Após isso, foi realizado um estudo de caso que
consistiu em replicar um aplicativo nativo utilizando uma ferramenta multiplataformas, para comprovar empiricamente as diferenças pesquisadas e descritas nas Seções~\ref{cap:referencialteorico} a~\ref{cap:comparativo} 
desse trabalho. Com base na experiência obtida no estudo de caso, foram feitas as considerações preliminares do trabalho e um planejamento do que poderá ser feito na continuação desse trabalho.

\begin{comment}
aqui eu trago o leitor pro nosso trabalho, ateh entao era coisas dos outros, aqui eu informo que meu trabalho vai ser esse

na introducao da metodologia, fazer esse resgate ao problema

qual a metodologia do estudo comparativo

em relacao ao que existe na literatura

estao discutindo isso

concluir dizendo o que vamos fazer de diferente, um estudo comparativo

dado isso nossa questao problema é: ....

NAO ESTAMOS VENDO A IMPORTANCIA DA RELACAO DA LINHA DE PRODUTO COM O CROSS PRO TCC 
ATEH ESTAMOS CORRELACIONAR, MAS EH MEIO FORCACAO DE BARRA
NAO FAZ PARTE DO FOCO DO NOSSO TRABALHO 
NAO SABEMOS COMO ABORDAR ESSE TEMA JA QUE NOSSA TEMA PARECE SER DIFERENTE DISSO

\end{comment}

\section{Trabalhos relacionados} \label{sec:trabalhosrelacionados}

Durante a pesquisa conduzida, foram encontrados alguns trabalhos relacionados com o tema abordado por este trabalho, como \citeonline{prezotto_estudo_2014}
que implementou um aplicativo utilizando a ferramenta multiplataforma \textit{Apache Cordova}. Concluiu-se que a abordagem multiplataformas 
não é melhor que a nativa, mas que há situações onde cada abordagem se encaixa melhor.

\citeonline{bezerra_desenvolvimento_2016} também fez a implementação de um aplicativo utilizando ferramenta multiplataforma, nesse caso, \textit{PhoneGap}. 
Concluiu-se que há a necessidade de avaliar os requisitos da aplicação a ser criada para se tomar a melhor decisão, sobre qual abordagem optar.
Cabe ressaltar, no entanto, que ambos os trabalhos não realizaram o desenvolvimento nativo para uma comparação mais precisa entre as duas formas de desenvolvimento
tratadas neste trabalho. 

\citeonline{charland_mobile_2011} faz um estudo comparativo teórico sobre desenvolvimento nativo e multiplataformas, no entanto,
não faz uma comparação prática entre as duas abordagens. Concluiu-se que o desenvolvimento híbrido é a solução mais provável no embate 
\textit{nativo vs cross}, no entanto, há de se avaliar as necessidades da aplicação e do negócio.

Em \citeonline{corral_ant_2012}, é feito um levantamento sobre potenciais vantagens e desvantagens da abordagem \textit{cross-platform} sob a 
perspectiva de três \textit{stakeholders} distintos: Usuário, Desenvolvedor e Provedor de Plataformas. Concluiu-se que os usuários são os principais
guias do mercado e as suas preferências é que devem definir o futuro do desenvolvimento móvel. No entanto, não foi feito nenhum estudo prático comparando
as duas abordagens alvo deste trabalho.

Embora a pesquisa tenha revelado que está sendo discutida qual a melhor abordagem para desenvolvimento móvel, os estudo feitos, em sua maioria, são de três a cinco 
anos atrás, o que invalida muitos dos argumentos usados para avaliar o desenvolvimento multiplataformas, pois as ferramentas e \textit{frameworks} evoluiram muito nesse
meio tempo. 

Não foram encontrados trabalhos recentes com a proposta de desenvolvimento de um mesmo \textit{app} utilizando as duas abordagens, para se ter um comparativo mais embasado das
características de cada abordagem, suas vantagens e desvantagens.
 
\begin{comment}
Falar que tem mt trabalho comparativo de nativo e cross, mas não refletem a atualidades, pois mudou muito em poucos anos.
%Por exemplo, em \citeonline{kassas_taxonomy_2015} é citado que
e nao ha um estudo comparativo
tenho 3 plataformas, o cross surgiu como uma solucao, e queremos confirmar se eh mesmo uma solucao
da pra confiar nessa abordagem pra todos os desenvolvimentos de apps?    
\end{comment}

\section{Seleção das plataformas} \label{sec:selecaodasplataformas}

Existem, atualmente, muitas ferramentas e \textit{frameworks} para o desenvolvimento de aplicações móveis multiplataformas. Dentre elas, a escolha do \textit{Ionic framework} se deu por alguns fatores, listados a seguir. 

\begin{enumerate}
    \item É um \textit{framework open source} e gratuito;
    \item Possui um ambiente de apoio gratuito e robusto, com muitas ferramentas e tutoriais disponíveis;
    \item Utiliza AngularJS e Cordova que são dois \textit{frameworks} estáveis, muito conhecidos, utilizados e mantidos por duas grandes empresas;
    \item Possui muitos \textit{plugins} disponíveis e uma comunidade muito ativa para resolução de problemas; %http://codepen.io/b00stup/post/why-ionic-framework
    \item Consegue criar uma aparência e usabilidade muito próximos dos dispositivos nativos; %http://www.joshmorony.com/8-reasons-why-im-glad-i-switched-to-the-ionic-framework/
    \item Possui uma arquitetura projetada para otimizar a performance nos dispositivos móveis; 
\end{enumerate}

\section{Seleção do projeto} \label{sec:selecaodoprojeto}

Para a realização do estudo de caso, foi escolhido o aplicativo Mini Farma, criado nativamente para a plataforma \textit{iOS}. A escolha se deu por alguns fatores, listados a seguir:

\begin{enumerate}
    \item É de propriedade de um dos autores deste trabalho;
    \item Explora algumas capacidades do dispositivo, que serão explicitadas mais adiante;
    \item Aplicativo pequeno e simples;
\end{enumerate}
 