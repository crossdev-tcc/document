\chapter{Metodologia} \label{cap:metodologia}
Neste capítulo são descritas as pesquisas feitas e os procedimentos realizados a 
fim de atingir os objetivos do trabalho, para detalhar como se deu o desenvolvimento do mesmo e, 
dessa forma, fornecer insumos para a sua correta reprodução futuramente por outros pesquisadores.

Esse trabalho baseia-se na comparação entre duas abordagens de desenvolvimento móvel, são elas, nativa e multiplataforma. 
Com isso, o objetivo principal é responder a seguinte questão problema:
\begin{center}
    \textit{Quais as vantagens e desvantagens do desenvolvimento multiplataforma de aplicações móveis em relação ao desenvolvimento nativo?}
\end{center}

A partir da questão feita, as duas hipóteses formuladas acerca do tema, são apresentadas e explicadas a seguir.

\begin{itemize}
    \item \textbf{H1}: As plataformas \textit{cross-development} estão evoluídas ao ponto de apresentarem mais vantagens do que desvantagens.
        \begin{itemize}
            \item As ferramentas para desenvolvimento \textit{cross-plataform} apresentam, atualmente, benefícios suficientes para serem consideradas uma opção viável ou talvez melhor que a abordagem nativa
            a depender da situação que envolve a aplicação a ser construída. 
        \end{itemize}
    \item \textbf{H2}: Ao longo do tempo, as possíveis desvantagens do desenvolvimento multiplataformas serão sanadas ou mitigadas.
        \begin{itemize}
            \item Observando a evolução das ferramentas multiplataformas ao longo dos últimos anos, pode-se perceber que não apresentam mais os gargalos citados na literatura.
        \end{itemize}
\end{itemize}

Para responder a questão problema será feita uma pesquisa sobre aplicações móveis e duas de suas abordagens de desenvolvimento: \textbf{nativa} e \textbf{multiplataformas}.
Para isso, serão utilizados artigos e materiais \textit{on-line} para definição de como é esse desenvolvimento e, com isso, será feita uma comparação entre os dois modos. 

Após o estudo e a definição de como se dá esse desenvolvimento específico de \textit{software},
será feita uma pesquisa de plataformas para desenvolvimento \textit{cross-platform} e então será selecionada uma ferramenta.% foi escolhida, no caso o \textit{Ionic}. 

Encerrada a fase de pesquisas, será realizado um estudo de caso, que consistirá em replicar um aplicativo nativo para plataforma \textit{iOS} utilizando a ferramenta selecionada, 
para comprovar empiricamente as diferenças pesquisadas e descritas no Capítulo~\ref{cap:referencialteorico} desse trabalho.

No decorrer da realização do estudo de caso, que será descrito mais adiante no 
Capítulo~\ref{cap:estudodecaso}, serão anotadas as diferenças na implementação das mesmas funcionalidades no ambiente nativo e no multiplataformas, e após isso, ambas as abordagens poderão ser comparadas
de maneira empírica com embasamento teórico oriundo das pesquisas feitas anteriormente.

Com base nas experiências obtidas no estudo de caso, serão feitas as considerações preliminares acerca dos resultados obtidos e um planejamento do que poderá ser feito na continuação desse trabalho.

\section{Trabalhos relacionados} \label{sec:trabalhosrelacionados}

Durante a pesquisa conduzida, foram encontrados alguns trabalhos relacionados com o tema abordado por este trabalho, como \citeonline{prezotto_estudo_2014}
que implementou um aplicativo utilizando a ferramenta multiplataforma \textit{Apache Cordova}. Concluiu-se que a abordagem multiplataformas 
não é melhor que a nativa, mas que há situações onde cada abordagem se encaixa melhor.

\citeonline{bezerra_desenvolvimento_2016} também fez a implementação de um aplicativo utilizando ferramenta multiplataforma, nesse caso, \textit{PhoneGap}. 
Concluiu-se que há a necessidade de avaliar os requisitos da aplicação a ser criada para se tomar a melhor decisão, sobre qual abordagem optar.
Cabe ressaltar, no entanto, que ambos os trabalhos não realizaram o desenvolvimento nativo para uma comparação mais precisa entre as duas formas de desenvolvimento
tratadas neste trabalho. 

\citeonline{charland_mobile_2011} faz um estudo comparativo teórico sobre desenvolvimento nativo e multiplataformas, no entanto,
não faz uma comparação prática entre as duas abordagens. Concluiu-se que o desenvolvimento híbrido é a solução mais provável no embate 
\textit{nativo vs cross}, no entanto, há de se avaliar as necessidades da aplicação e do negócio.

Em \citeonline{corral_ant_2012}, é feito um levantamento sobre potenciais vantagens e desvantagens da abordagem \textit{cross-platform} sob a 
perspectiva de três \textit{stakeholders} distintos: Usuário, Desenvolvedor e Provedor de Plataformas. Concluiu-se que os usuários são os principais
guias do mercado e as suas preferências é que devem definir o futuro do desenvolvimento móvel. No entanto, não foi feito nenhum estudo prático comparando
as duas abordagens alvo deste trabalho.

Embora a pesquisa tenha revelado que está sendo discutida qual a melhor abordagem para desenvolvimento móvel, os estudo feitos, em sua maioria, são de três a cinco 
anos atrás, o que invalida muitos dos argumentos usados para avaliar o desenvolvimento multiplataformas, pois as ferramentas e \textit{frameworks} evoluiram muito nesse
meio tempo. 

Não foram encontrados trabalhos recentes com a proposta de desenvolvimento de um mesmo \textit{app} utilizando as duas abordagens, para se ter um comparativo mais embasado das
características de cada abordagem, suas vantagens e desvantagens.
 
\begin{comment}
Falar que tem mt trabalho comparativo de nativo e cross, mas não refletem a atualidades, pois mudou muito em poucos anos.
%Por exemplo, em \citeonline{kassas_taxonomy_2015} é citado que
e nao ha um estudo comparativo
tenho 3 plataformas, o cross surgiu como uma solucao, e queremos confirmar se eh mesmo uma solucao
da pra confiar nessa abordagem pra todos os desenvolvimentos de apps?    
\end{comment}

\section{Planejamento do Estudo de Caso} \label{section:planejamentoestudodecaso}

Nas subseções a seguir, são apresentados os critérios usados pelos autores para a seleção do projeto a ser recriado e da ferramenta a ser utilizada para o estudo de caso, assim como um planejamento
do desenvolvimento do projeto.

\subsection{Seleção do projeto} \label{subsection:selecaodoprojeto}

Para a realização do estudo de caso, foi escolhido o aplicativo Mini Farma, criado nativamente para a plataforma \textit{iOS}. A escolha se deu por alguns fatores, listados a seguir:

\begin{itemize}
    \item É de propriedade de um dos autores deste trabalho, o que facilita na obtenção e no entendimento do código;
    \item Explora funcionalidades nativas do dispositivo, como localização por exemplo, que serão explicitadas mais adiante;
    \item É um aplicativo pequeno e simples, o que o torna ideal para ser refeito dentro do tempo de execução do trabalho;
\end{itemize}

\subsection{Seleção das plataformas} \label{subsection:selecaodasplataformas}

Existem, atualmente, muitas ferramentas e \textit{frameworks} para o desenvolvimento de aplicações móveis multiplataformas. Dentre elas, a escolha do \textit{Ionic framework} se deu por alguns fatores, listados a seguir. 

\begin{itemize}
    \item É um \textit{framework open source} e gratuito;
    \item Possui um ambiente de apoio gratuito e robusto, com muitas ferramentas e tutoriais disponíveis;
    \item Utiliza \textit{AngularJS} e \textit{Cordova} que são dois \textit{frameworks} estáveis, muito conhecidos e utilizados e mantidos por duas grandes empresas;
    \item Possui muitos \textit{plugins} disponíveis e uma comunidade muito ativa para resolução de problemas; %http://codepen.io/b00stup/post/why-ionic-framework
    \item Consegue criar uma aparência e usabilidade muito próximos das plataformas nativas; %http://www.joshmorony.com/8-reasons-why-im-glad-i-switched-to-the-ionic-framework/
    \item Possui uma arquitetura projetada para otimizar a performance nos dispositivos móveis; 
\end{itemize}
 
\subsection{Planejamento do desenvolvimento} \label{subsection:planejamentodesenvolvimento}

Após definidos o projeto e a ferramenta que serão utilizadas, o próximo passo é planejar como será feito o desenvolvimento do \textit{app} multiplataformas. 
A replicação do projeto passou por algumas etapas, citadas e detalhadas a seguir.

\begin{itemize}
    \item Configuração do \textit{framework Ionic} e suas dependências nas máquinas dos desenvolvedores;
    \item Definição e implantação de ambiente de desenvolvimento igual para ambos os desenvolvedores;
    \item Familiarização com \textit{framework Ionic} e seu ambiente por meio de criação de projetos teste e tutoriais;
    \item Familiarização com \textit{AngularJS} e sua arquitetura; 
    \item Separação do aplicativo Mini Farma nas seguintes funcionalidades principais a serem recriadas:
        \begin{itemize}
            \item \textbf{Listagem de Remédios e Alertas};
            \item \textbf{Cadastro de Remédios}: sendo necessário o acesso a câmera fotográfica e galeria de fotos do dispositivo;
            \item \textbf{Cadastro de Farmácias}: sendo necessário o acesso à localização do dispositivo e ligação de voz;
            \item \textbf{Cadastro de Alertas}: sendo necessário o acesso à central de notificações locais do dispositivo;
        \end{itemize}
        \textit{* Todas as funcionalidades exigem acesso ao banco de dados local SQLite criado para o aplicativo}
    \item Relatar observações acerca das diferenças, dificuldades, facilidades, vantagens e desvantagens;
\end{itemize}