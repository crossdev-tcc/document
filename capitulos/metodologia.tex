\chapter{Metodologia} \label{cap:metodologia}
%Nesta seção é apresentada a metodologia utilizada para o desenvolvimento desse trabalho.
Para a realização deste trabalho foi feita, inicialmente, uma pesquisa sobre desenvolvimento de aplicações móveis. 
Para isso, foram utilizados artigos e materiais \textit{on-line} para definição de como é esse desenvolvimento
de maneira nativa e multiplataformas e, com isso, foi feita uma comparação entre os dois modos. 


Após o estudo e a definição de como se dá esse desenvolvimento específico de \textit{software},
foi feita uma pesquisa de plataformas para desenvolvimento \textit{cross-platform} e posteriormente uma ferramenta foi escolhida. Após isso, foi realizado um estudo de caso que
consistiu em replicar um aplicativo nativo utilizando uma ferramenta multiplataformas, para comprovar empiricamente as diferenças pesquisadas e descritas na seção 2 desse trabalho. 
Com base na experiência obtida no estudo de caso, foram feitas as considerações preliminares
do trabalho e um planejamento do que poderá ser feito em trabalhos futuros.
%\section{Planejamento da pesquisa} \label{sec:planejamento}


aqui eu trago o leitor pro nosso trabalho, ateh entao era coisas dos outros, aqui eu informo que meu trabalho vai ser esse

na introducao da metodologia, fazer esse resgate ao problema

qual a metodologia do estudo comparativo

em relacao ao que existe na literatura

estao discutindo isso

concluir dizendo o que vamos fazer de diferente, um estudo comparativo

dado isso nossa questao problema é: ....

NAO ESTAMOS VENDO A IMPORTANCIA DA RELACAO DA LINHA DE PRODUTO COM O CROSS PRO TCC 
ATEH ESTAMOS CORRELACIONAR, MAS EH MEIO FORCACAO DE BARRA
NAO FAZ PARTE DO FOCO DO NOSSO TRABALHO 
NAO SABEMOS COMO ABORDAR ESSE TEMA JA QUE NOSSA TEMA PARECE SER DIFERENTE DISSO
A UNICA FORMA DE ENCAIXAR LINHA DE PRODUTO NO TCC EH CRIAR UM SISTEMA PARA CRIACAO DE LINHA DE PRODUTOS DE APPS

\section{Trabalhos relacionados} \label{sec:trabalhosrelacionados}
Falar que tem mt trabalho comparativo de nativo e cross, mas não refletem a atualidades, pois mudou muito em poucos anos.

Nesse artigo, por exemplo, diz q cross não tem acesso a camera http://www.sciencedirect.com/science/article/pii/S2090447915001276

e nao ha um estudo comparativo


%o dev de apps moveis requer outros requisitos em relacao a outros sistemas... blablabla

tenho 3 plataformas, o cross surgiu como uma solucao, e queremos confirmar se eh mesmo uma solucao

da pra confiar nessa abordagem pra todos os desenvolvimentos de apps?


%entender sobre linha de produto, pra ver se cabe
%mostrar vantagens numa pesquisa sobre linha de produto no tcc2
%linha de produto é um processo
%num ambiente cross tem uma vantagem, de poder adaptar meu app para varias plataformas
%nao precisa se aprofundar nisso, só na proposta mesmo, dizer que vai ser continuado no tcc2
%e isso eh uma coisa de mais uma vantagem para o desenvolvimento cross
%o ciclo de manutencao/evolucao de um app eh mais ciclo
%bem objetivamente, nao se perder no blablabla

 
\section{Questão Problema} \label{sec:questaoproblema}

A questão problema é apresentada a seguir.

\begin{itemize}
    \item Quais as vantagens e desvantagens do desenvolvimento multiplataforma de aplicações móveis em relação ao desenvolvimento nativo?
    \begin{itemize}
        \item \textbf{H1}: Atualmente as plataformas de \textit{cross-development} estão evoluídas ao ponto de apresentarem mais vantagens do que desvantagens.
        \item \textbf{H2}: O desenvolvimento multiplataforma melhor se adequa a enegenharia de linha de produto de software (TCC2).
    \end{itemize}

    %\item ------------------
    %\item \textcolor{red}{\textit{\textbf{ COISAS CONVERSADAS COM O PROF}}}
    %\item ------------------
    %\item dev com cross tem certos gargalos, funciona para projetos mais simples, nossa aposta inicial
    %\item nao precisa mais falar isso, foi inicial nossa mas ja viu que nao eh bem assim 
    %\item dia 15/06 uma versao pronta pra prof revisar
    %\item até dia 22/06 no portal
    %\item professor prefere deixar as defesas pra julho, pois ele tem muitas coisa pra fazer
    %\item validar a arquitetura e achar os gargalos
    %\item o suficiente para escrever
    %\item nao ha trabalhos relacionados
    %\item TITULO DO TRABALHO: Estudo comparativo entre o desenvolvimento de aplicativos móveis utilizando plataformas nativas e multiplataformas
     
\end{itemize}

\section{Seleção das plataformas} \label{sec:selecaodasplataformas}

Existem, atualmente, muitas ferramentas e \textit{frameworks} para o desenvolvimento de aplicações móveis multiplataformas. Dentre elas, a escolha do \textit{Ionic framework} se deu por alguns fatores, listados a seguir. 

\begin{enumerate}
    \item É um \textit{framework open source} e gratuito;
    \item Possui um ambiente de apoio gratuito e robusto, com muitas ferramentas e tutoriais disponíveis;
    \item Utiliza AngularJS e Cordova que são dois \textit{frameworks} estáveis, muito conhecidos, utilizados e mantidos por duas grandes empresas;
    \item Possui muitos \textit{plugins} disponíveis e uma comunidade muito ativa para resolução de problemas; %http://codepen.io/b00stup/post/why-ionic-framework
    \item Consegue criar uma aparência e usabilidade muito próximos dos dispositivos nativos; %http://www.joshmorony.com/8-reasons-why-im-glad-i-switched-to-the-ionic-framework/
    \item Possui uma arquitetura projetada para otimizar a performance nos dispositivos móveis; 
\end{enumerate}

\section{Seleção do projeto} \label{sec:selecaodoprojeto}

Para a realização do estudo de caso, foi escolhido o aplicativo Mini Farma, criado nativamente para a plataforma \textit{iOS}. A escolha se deu por alguns fatores, listados a seguir:

\begin{enumerate}
    \item É de propriedade de um dos autores deste trabalho;
    \item Explora algumas capacidades do dispositivo, que serão explicitadas mais adiante;
    \item Aplicativo pequeno e simples;
\end{enumerate}
 