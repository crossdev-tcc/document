\chapter{Metodologia} \label{cap:metodologia}
Neste capítulo são descritas as pesquisas feitas e os procedimentos realizados a 
fim de atingir os objetivos do trabalho, para detalhar como se deu o desenvolvimento do mesmo e, 
dessa forma, fornecer insumos para a sua correta reprodução futuramente por outros pesquisadores.

Este trabalho baseia-se na comparação entre duas abordagens de desenvolvimento móvel, são elas, nativa e multiplataforma. 
Com isso, o objetivo principal é responder a seguinte questão problema:
\begin{center}
    \textit{Quais as vantagens e desvantagens do desenvolvimento multiplataforma de aplicações móveis em relação ao desenvolvimento nativo?}
\end{center}

A partir da questão feita, as duas hipóteses formuladas acerca do tema, são apresentadas e explicadas a seguir.

\begin{itemize}
    \item \textbf{H1}: As ferramentas multiplataforma estão evoluídas ao ponto de apresentarem mais vantagens do que desvantagens.
        \begin{itemize}
            \item As ferramentas para desenvolvimento multiplataforma apresentam, atualmente, benefícios suficientes para serem consideradas uma opção viável ou talvez melhor que a abordagem nativa
            a depender da situação que envolve a aplicação a ser construída. 
        \end{itemize}
    \item \textbf{H2}: Ao longo do tempo, as possíveis desvantagens do desenvolvimento multiplataforma serão sanadas ou mitigadas.
        \begin{itemize}
            \item Observando a evolução das ferramentas multiplataforma ao longo dos últimos anos, pode-se perceber que não apresentam mais os gargalos citados na literatura.
        \end{itemize}
\end{itemize}

Para responder a questão problema será feita uma pesquisa sobre aplicações móveis e duas de suas abordagens de desenvolvimento: \textbf{nativa} e \textbf{multiplataforma}.
Para isso, serão utilizados artigos e materiais \textit{on-line} para definição de como é esse desenvolvimento e, com isso, será feita uma comparação entre os dois modos. 

Após o estudo e a definição de como se dá esse desenvolvimento específico de \textit{software},
será feita uma pesquisa de plataformas para desenvolvimento multiplataforma e então será selecionada uma ferramenta. 

Encerrada a fase de pesquisas, será realizado um primeiro estudo de uso, que consistirá em replicar um aplicativo nativo para plataforma iOS utilizando a ferramenta selecionada, 
para comprovar empiricamente as diferenças pesquisadas e descritas no Capítulo~\ref{cap:referencialteorico} deste trabalho.

No decorrer da realização do estudo, que será descrito mais adiante no 
Capítulo~\ref{cap:estudodecaso}, serão anotadas as diferenças na implementação das mesmas funcionalidades no ambiente nativo e no multiplataforma, e após isso, ambas as abordagens poderão ser comparadas
de maneira empírica com embasamento teórico oriundo das pesquisas feitas anteriormente.

Com base nas experiências obtidas no estudo realizado, serão feitas as considerações preliminares acerca dos resultados obtidos e um planejamento do que poderá ser feito na continuação deste trabalho.

\section{Trabalhos relacionados} \label{sec:trabalhosrelacionados}

Durante a pesquisa conduzida, foram encontradas pesquisas relacionadas com o tema abordado por este trabalho, como \citeonline{prezotto_estudo_2014}
que implementou um aplicativo utilizando a ferramenta multiplataforma \textit{Apache Cordova}. A conclusão da pesquisa foi, 
que a abordagem multiplataforma não é melhor que a nativa, mas que há situações onde cada abordagem se encaixa melhor.

\citeonline{bezerra_desenvolvimento_2016} também fez a implementação de um aplicativo utilizando uma ferramenta multiplataforma, nesse caso, o PhoneGap. 
O autor concluiu que há a necessidade de avaliar os requisitos da aplicação a ser criada para se tomar a melhor decisão, sobre qual abordagem optar.
Cabe ressaltar, no entanto, que ambos os trabalhos não realizaram o desenvolvimento nativo para uma comparação mais precisa entre as duas formas de desenvolvimento
tratadas neste trabalho. 

\citeonline{charland_mobile_2011} faz um estudo comparativo teórico sobre desenvolvimento nativo e multiplataforma, no entanto,
não faz uma comparação prática entre as duas abordagens. A conclusão que o autor chega é que o desenvolvimento híbrido é a solução mais provável no embate 
\textit{nativo vs multiplataforma}, no entanto, há de se avaliar as necessidades da aplicação e do negócio.

Em \citeonline{corral_ant_2012}, é feito um levantamento sobre potenciais vantagens e desvantagens da abordagem multiplataforma sob a 
perspectiva de três \textit{stakeholders} distintos: Usuário, Desenvolvedor e Provedor de Plataformas. O autor chegou à conclusão que os usuários são os principais
guias do mercado e as suas preferências é que devem definir o futuro do desenvolvimento móvel. No entanto, não foi feito nenhum estudo prático comparando
as duas abordagens alvo deste trabalho.

Embora a pesquisa tenha revelado que está sendo discutida qual a melhor abordagem para o desenvolvimento móvel, os estudo feitos, em sua maioria, são de três a cinco 
anos atrás, fazendo-se necessárias novas avaliações do desenvolvimento multiplataforma, devido à evolução das ferramentas e \textit{frameworks} neste meio tempo.
%o que invalida muitos dos argumentos usados para avaliar
%com isso este torna-se um momento mais adequado para pesquisar o
Não foram encontrados trabalhos recentes com a proposta de desenvolvimento de um mesmo \textit{app} utilizando as duas abordagens, para se ter um comparativo mais embasado das
características de cada abordagem, suas vantagens e desvantagens.
 
\begin{comment}
Falar que tem mt trabalho comparativo de nativo e cross, mas não refletem a atualidades, pois mudou muito em poucos anos.
%Por exemplo, em \citeonline{kassas_taxonomy_2015} é citado que
e nao ha um estudo comparativo
tenho 3 plataformas, o cross surgiu como uma solucao, e queremos confirmar se eh mesmo uma solucao
da pra confiar nessa abordagem pra todos os desenvolvimentos de apps?    
\end{comment}

\section{Planejamento do Exemplo de Uso} \label{section:planejamentoestudodecaso}

Nas subseções a seguir, são apresentados os critérios usados pelos autores para a seleção do projeto a ser recriado e da ferramenta a ser utilizada para um primeiro estudo de uso, assim como um planejamento
do desenvolvimento do projeto.

\subsection{Seleção do projeto} \label{subsection:selecaodoprojeto}

Para a realização do exemplo de uso, foi escolhido o aplicativo Mini Farma, criado nativamente para a plataforma iOS. A escolha se deu por alguns fatores, listados a seguir:

\begin{itemize}
    \item É de propriedade de um dos autores deste trabalho, o que facilita na obtenção e no entendimento do código;
    \item Explora funcionalidades nativas do dispositivo, como localização, que serão explicitadas mais adiante;
    \item É um aplicativo pequeno e simples, o que o torna ideal para ser refeito dentro do tempo de execução do trabalho;
\end{itemize}

\subsection{Seleção das plataformas} \label{subsection:selecaodasplataformas}

Existem, atualmente, muitas ferramentas e \textit{frameworks} para o desenvolvimento de aplicações móveis multiplataforma. Dentre elas, a escolha do Ionic se deu por alguns fatores, listados a seguir. 

\begin{itemize}
    \item É um \textit{framework} sob uma licença de \textit{software} livre;
    \item Possui um ambiente de apoio gratuito e robusto, com muitas ferramentas e tutoriais disponíveis;
    \item Utiliza AngularJS e Cordova que são dois \textit{frameworks} estáveis, muito conhecidos e utilizados e mantidos por duas grandes empresas;
    \item Possui muitos \textit{plugins} disponíveis e uma comunidade muito ativa para resolução de problemas; %http://codepen.io/b00stup/post/why-ionic-framework
    \item Consegue criar aparência e usabilidade próximas às das plataformas nativas; %http://www.joshmorony.com/8-reasons-why-im-glad-i-switched-to-the-ionic-framework/
    \item Possui uma arquitetura projetada para otimizar a performance nos dispositivos móveis; 
\end{itemize}
 
\subsection{Planejamento do desenvolvimento} \label{subsection:planejamentodesenvolvimento}

Após definidos o projeto e a ferramenta que serão utilizadas, o próximo passo é planejar como será feito o desenvolvimento do \textit{app} multiplataforma. 
A replicação do projeto passou por algumas etapas, citadas e detalhadas a seguir.

\begin{itemize}
    \item Configuração do \textit{framework} Ionic e suas dependências nas máquinas dos desenvolvedores;
    \item Definição e implantação de ambiente de desenvolvimento igual para ambos os desenvolvedores;
    \item Familiarização com \textit{framework} Ionic e seu ambiente por meio de criação de projetos teste e tutoriais;
    \item Familiarização com AngularJS e sua arquitetura; 
    \item Separação do aplicativo Mini Farma nas suas principais funcionalidades a serem recriadas:
    \item Relato de observações acerca das diferenças, dificuldades, facilidades, vantagens e desvantagens;
\end{itemize}