\chapter{Metodologia} \label{metodologia}

Nesta seção é apresentada a metodologia utilizada para o desenvolvimento desse trabalho.

\section{Planejamento da pesquisa} \label{sec:planejamento}

Para a realização deste trabalho foi feita uma pesquisa sobre desenvolvimento de aplicativos móveis. Para isso, foram utilizados artigos e materiais \textit{on-line} para definição de como é o desenvolvimento
de aplicações móveis de maneira nativa e de maneira multiplataformas e, com isso, foi feita uma comparação entre os dois modos. Após a definição de como se dá esse desenvolvimento específico de \textit{software},
foi feito um estudo de caso para comprovar empiricamente as diferenças pesquisadas e descritas na pesquisa realizada. Com base na experiência tida no estudo de caso, foram feitas as considerações preliminares
do trabalho.

\subsection{Hipóteses} \label{subsec:hipoteses}

H1: Manutenção de aplicativos nativos é mais fácil que multiplataformas. 
(Demora na adaptação de frameworks multiplataformas para as atualizações lançadas para as plataformas nativas, ou seja, quando há uma atualização de ambiente ou linguagem nativa,
os frameworks demoram para poder integrar as novas funcionalidades)


H2: Desenvolvimento multiplataformas pode ser visto como uma linha de produtos.


\subsection{Seleção das plataformas} \label{subsec:selecaodasplataformas}

A escolha do \textit{Ionic framework} se deu por alguns fatores, listados a seguir. 
\begin{enumerate}
    \item Novo, open source, gratuito,
    \item Ambiente de apoio gratuito e robusto
    \item Utiliza AngularJS e Cordova que são dois frameworks estáveis, muito conhecidos e utilizados
    \item Muitos plugins disponíveis e comunidade muito ativa para resolução de problemas %http://codepen.io/b00stup/post/why-ionic-framework
    \item Framework que consegue imitar melhor a aparencia e usabilidade dos dispositivos nativos %http://www.joshmorony.com/8-reasons-why-im-glad-i-switched-to-the-ionic-framework/
\end{enumerate}
 
\begin{comment}
Porque Ionic? Porque não, outro? Quais os criterios usados para escolher o Ionic?
Ionic é open source, gratuito, muito focado em performance e possui html e css otimizados para dispositivos móveis.
\end{comment}

\subsection{Seleção do projeto} \label{subsec:selecaodoprojeto}

A escolha do aplicativo Mini Farma se deu por alguns fatores, listados a seguir:

\begin{enumerate}
    \item É de propriedade de um dos autores deste trabalho;
    \item Explora algumas capacidades do dispositivo, que serão explicitadas mais adiante;
    \item Aplicativo pequeno e simples;
\end{enumerate}
 