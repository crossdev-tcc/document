\chapter{Metodologia} \label{metodologia}
%Nesta seção é apresentada a metodologia utilizada para o desenvolvimento desse trabalho.
Para a realização deste trabalho foi feita, inicialmente, uma pesquisa sobre desenvolvimento de aplicações móveis. 
Para isso, foram utilizados artigos e materiais \textit{on-line} para definição de como é esse desenvolvimento
de maneira nativa e multiplataformas e, com isso, foi feita uma comparação entre os dois modos. 


Após o estudo e a definição de como se dá esse desenvolvimento específico de \textit{software},
foi feita uma pesquisa de plataformas para desenvolvimento \textit{cross-platform} e posteriormente uma ferramenta foi escolhida. Após isso, foi realizado um estudo de caso que
consistiu em replicar um aplicativo nativo utilizando uma ferramenta multiplataformas, para comprovar empiricamente as diferenças pesquisadas e descritas na seção 2 desse trabalho. 
Com base na experiência obtida no estudo de caso, foram feitas as considerações preliminares
do trabalho e um planejamento do que poderá ser feito em trabalhos futuros.
%\section{Planejamento da pesquisa} \label{sec:planejamento}
\section{Hipóteses/Questão de Pesquisa} \label{subsec:hipoteses}

As hipóteses/questões de pesquisa são apresentadas a seguir.

\begin{itemize}
    \item \textbf{H1}: Manutenção de aplicativos nativos é mais fácil que multiplataformas.
    %(Demora na adaptação de frameworks multiplataformas para as atualizações lançadas para as plataformas nativas, ou seja, quando há uma atualização de ambiente ou linguagem nativa, 
    %os frameworks demoram para poder integrar as novas funcionalidades)
    %depende muito da criacao de plugins de terceiros para poder implementar as funcionalidades nativas 
    \item \textbf{H2}: Desenvolvimento multiplataformas pode ser visto como uma linha de produtos.
    \item \textbf{Q1}: O desenvolvimento multiplataformas apresenta mais desvantagens do que o nativo?
    \item \textbf{Q2}: A manutenção de aplicativos multiplataformas é mais fácil do que a dos nativos?
    \item \textbf{Q3}: A evolução de aplicativos multiplataformas é mais difícil do que a dos nativos? 
\end{itemize}

\section{Seleção das plataformas} \label{subsec:selecaodasplataformas}

Existem, atualmente, muitas ferramentas e \textit{frameworks} para o desenvolvimento de aplicações móveis multiplataformas. Dentre elas, a escolha do \textit{Ionic framework} se deu por alguns fatores, listados a seguir. 

\begin{enumerate}
    \item É um \textit{framework open source} e gratuito;
    \item Possui um ambiente de apoio gratuito e robusto, com muitas ferramentas e tutoriais disponíveis;
    \item Utiliza AngularJS e Cordova que são dois \textit{frameworks} estáveis, muito conhecidos, utilizados e mantidos por duas grandes empresas;
    \item Possui muitos \textit{plugins} disponíveis e uma comunidade muito ativa para resolução de problemas; %http://codepen.io/b00stup/post/why-ionic-framework
    \item Consegue criar uma aparência e usabilidade muito próximos dos dispositivos nativos; %http://www.joshmorony.com/8-reasons-why-im-glad-i-switched-to-the-ionic-framework/
    \item Possui uma arquitetura projetada para otimizar a performance nos dispositivos móveis; 
\end{enumerate}

\section{Seleção do projeto} \label{subsec:selecaodoprojeto}

Para a realização do estudo de caso, foi escolhido o aplicativo Mini Farma, criado nativamente para a plataforma \textit{iOS}. A escolha se deu por alguns fatores, listados a seguir:

\begin{enumerate}
    \item É de propriedade de um dos autores deste trabalho;
    \item Explora algumas capacidades do dispositivo, que serão explicitadas mais adiante;
    \item Aplicativo pequeno e simples;
\end{enumerate}
 