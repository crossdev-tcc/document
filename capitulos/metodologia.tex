\chapter{Metodologia} \label{metodologia}

Texto introdutório da seção

\section{Planejamento da pesquisa} \label{sec:planejamento}

O planejamento

\subsection{Hipóteses} \label{subsec:hipoteses}

As hipóteses

\subsection{Seleção das plataformas} \label{subsec:selecaodasplataformas}

A escolha do \textit{Ionic framework} se deu por alguns fatores, listados a seguir. 
\begin{enumerate}
    \item Novo, open source, gratuito,
    \item Ambiente de apoio gratuito e robusto
    \item Utiliza AngularJS e Cordova que são dois frameworks estáveis, muito conhecidos e utilizados
    \item Muitos plugins disponiveis e comunidade muito ativa para resolução de problemas %http://codepen.io/b00stup/post/why-ionic-framework
    \item Framework que consegue imitar melhor a aparencia e usabilidade dos dispositivos nativos %http://www.joshmorony.com/8-reasons-why-im-glad-i-switched-to-the-ionic-framework/
\end{enumerate}
 
\begin{comment}
Porque Ionic? Porque não, outro? Quais os criterios usados para escolher o Ionic?
Ionic é open source, gratuito, muito focado em performance e possui html e css otimizados para dispositivos móveis.
\end{comment}

\subsection{Seleção do projeto} \label{subsec:selecaodoprojeto}

A escolha do aplicativo Mini Farma se deu por alguns fatores, listados a seguir:

\begin{enumerate}
    \item É de propriedade de um dos autores deste trabalho;
    \item Explora algumas capacidades do dispositivo, que serão explicitadas mais adiante;
    \item Aplicativo pequeno e simples;
\end{enumerate}
 