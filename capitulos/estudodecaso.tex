\chapter{Estudo de Caso} \label{cap:estudodecaso}

Com o intuito de comparar o desenvolvimento de aplicações móveis utilizando ferramentas nativas e multiplataformas, 
foi realizado um estudo de caso. Para isso, um aplicativo feito em iOS 
nativo foi recriado utilizando o \textit{framework} Ionic. A seguir o aplicativo escolhido é detalhado em termos de funcionalidades e recursos utilizados. 

\section{Descrição do projeto selecionado} \label{sec:descricaodoprojeto}

Mini Farma é um aplicativo criado inicialmente para a plataforma iOS que serve para controle dos medicamentos que as 
pessoas possuem em casa em suas ``farmacinhas'' particulares.

\begin{figure}[h]
  \centering
    \includegraphics[width=0.5\textwidth]{minifarma}
    \caption[Tela inicial do aplicativo Mini Farma]{ Tela inicial do aplicativo Mini Farma. Fonte: Elaborado pelos autores.}
	\label{fig:minifarma}
\end{figure}

O aplicativo utiliza recursos nativos do sistema, os quais são descritos a seguir. 
\begin{itemize}
	\item \textbf{Localização geográfica}: Define a posição geográfica do dispositivo que o usuário está utilizando para salvar a 
    localização da farmácia onde ele comprou um medicamento para, caso seja necessário, explicar para alguém onde a 
    farmácia fica. Também é possível com base nessa localização da farmácia e do usuário, traçar uma rota para levá-lo diretamente à 
    farmácia em questão;
	\item \textbf{Notificações locais}: Diferente das notificações \textit{Push} as notificações locais são criadas e agendadas na central 
    de notificações do dispositivo e o sistema se encarrega de entregá-las corretamente de acordo com parâmentros definidos pelo aplicativo. 
    No caso do Mini Farma, as notificações são usadas para lembrar o usuário, na data e hora corretas, de que o mesmo deve tomar seus medicamentos;
	\item \textbf{Ligação}: As notificações podem conter ações que executam um determinado bloco de código dentro do aplicativo. No Mini Farma, 
    uma das notificações possíveis é a de aviso de pouca quantidade ou remédio esgotado, no caso do medicamento ter acabado. 
    Quando essa notificação é enviada, pode ser feita uma ligação diretamente pela ação da notificação para o número da 
    farmácia cadastrado no aplicativo. Dessa forma, o usuário pode solicitar uma nova quantidade de medicamento diretamente com a farmácia. 
	\item \textbf{Câmera}: Para facilitar a identificação dos medicamentos, é possível tirar uma foto com a câmera do dispositivo para 
    cada remédio cadastrado. Além de uma foto para o medicamento em si, é possível tirar uma foto da receita dele, caso haja;
    \item \textbf{Rolo de câmera}: Se o usuário já tiver uma foto que o ajude a identificar o seu medicamento ou da receita do mesmo salva 
    no rolo de câmera, é possível escolher a foto sem precisar tirar uma nova;
\end{itemize}

\textit{* Vale ressaltar, que com excessão da Ligação, todos os outros recursos nativos do sistema, precisam necessariamente serem autorizados pelo
usuário para funcionar. Caso o usuário não autorize, o aplicativo ficará com funcionalidades reduzidas.} 

Como banco de dados foi usado o SQLite, por meio do \textit{framework} externo e \textit{open-source FMDB}, 
disponível no \href{https://github.com/ccgus/fmdb}{Github};

A arquitetura do sistema foi criada com base no padrão de projeto \textit{MVC}, possuindo uma camada \textit{DAO} para comunicação 
com o banco de dados. No entanto, o \textit{MVC} tradicional difere em alguns aspectos do \textit{MVC} usado no iOS. 

No iOS o \textit{MVC} trás uma controladora ligada à classe de \textit{view}, que é chamado de \textit{ViewController}, que é responsável por 
criar a ponte entre a interação do usuário pela \textit{view} com as classes modelos. Dessa forma, as \textit{ViewControllers} têm a 
responsabilidade de repassar as entradas do usuário para a modelo e controlar a \textit{view} para apresentar os resultados que a modelo retornar.

Diferente do \textit{MVC} tradicional, no qual a controladora apenas repassa as informações que estão entrando na fronteira da aplicação, 
as \textit{ViewControllers} têm, além dessa responsabilidade, também a responsabilidade de instanciar e gerenciar a \textit{view} no tocante
a organização dos elementos e apresentação das informações.

Outra observação sobre as diferenças entre os dois \textit{MVCs}, é que no \textit{MVC} tradicional pode haver uma controladora apenas 
para todas as modelos ou uma controladora por modelo, mas nenhuma ligada diretamente a uma \textit{view}, no entanto, no iOS há uma 
\textit{ViewController} por \textit{view}, podendo haver ainda uma controladora relacionada com cada modelo conforme o padrão tradicional do \textit{MVC}.

\begin{figure}[h]
  \centering
    \includegraphics[width=1.0\textwidth]{mvc}
    \caption[Padrão \textit{Model-View-Controller}]{ Padrão \textit{Model-View-Controller}. Fonte: Apple Documentation.}
	\label{fig:mvc}
\end{figure}

\subsection{Ambiente de desenvolvimento} \label{subsec:ambientedesenvolvimento}
Nesta seção é apresentado todo o ambiente de desenvolvimento do \textit{app} Mini Farma quando foi feito originalmente para a plataforma iOS. 
O projeto foi executado contando com uma equipe de dois integrantes com conhecimentos intermediários na plataforma iOS. 
Os detalhes técnicos são listados a seguir.
\begin{itemize}
    \item \textbf{Máquinas}: Dois MacBook Pro Retina 13", processador \textit{Intel Core i5}, 8 GB de \textit{RAM};
    \item \textbf{Sistema Operacional das máquinas}: \textit{Mac OS X Yosemite} v10.10.5;
    \item \textbf{Sistema Operacional alvo do aplicativo}: iOS 8;
    \item \textbf{\textit{IDE}}: \textit{Xcode} v6.4;
    \item \textbf{Linguagem de Programação}: \textit{Swift} v1.3;
    \item \textbf{Banco de dados}: SQLite v3.8.10.2;
    \item \textbf{Ambiente de suporte}: \textit{Plugin SQLite Manager} para \textit{Mozilla Firefox}, para criação do banco de dados;
    %\item \textbf{Tempo de desenvolvimento}: Ambos os integrantes trabalharam no projeto quatro horas por dia de segunda à sexta por três semanas;
\end{itemize}

\section{Desenvolvimento multiplataformas do projeto} \label{sec:desenvolvimentomulti}

A partir do aplicativo feito em iOS, foi construída uma versão utilizando o Ionic para as plataformas Android e iOS.
Para a criação dessa versão do \textit{app}, a equipe foi alterada, mas permaneceu com dois integrantes, que são os autores deste trabalho.
No entanto, os integrantes não possuíam qualquer conhecimento prévio nos \textit{frameworks} Ionic, AngularJS e Cordova
e apenas possuíam um conhecimento básico em \textit{HTML}, \textit{CSS} e \textit{JavaScript}. Os detalhes técnicos são listados a seguir.
   
\begin{itemize}
    \item \textbf{Máquinas}: Dois MacBook Pro Retina 13", processador \textit{Intel Core i5}, 8 GB de \textit{RAM};
    \item \textbf{Sistema Operacional das máquinas}: \textit{Mac OS X El Capitan} v10.11.5;
    \item \textbf{Sistema Operacional alvo do aplicativo}: iOS 9.3 e Android 6.0.0;
    \item \textbf{\textit{IDE}}: \textit{WebStorm} v2016.1.1;
    \item \textbf{\textit{Frameworks}}:
    \begin{itemize}
        \item \textbf{Ionic}: \textit{Copenhagen} v1.2.4*;
        \item \textbf{AngularJS}: \textit{foam-acceleration} v1.4.3*;
        
        * Durante a execução deste trabalho, já haviam sido disponibilizadas as versões 2 do Ionic e do AngularJS, no entanto,
        os autores optaram por não utilizá-las por ainda se tratarem de versões \textit{beta} e portanto instáveis, o que poderia 
        impactar negativamente no desenvolvimento do projeto.
        
        \item \textbf{Cordova}: v6.0.0;
    \end{itemize}
    \item \textbf{Linguagem de Programação}: \textit{JavaScript} v1.7. 
    \begin{itemize}
        \item \textit{HTML} 5 e \textit{CSS} 3 foram usados paralelamente para a criação da interface gráfica do aplicativo;
    \end{itemize}
    \item \textbf{Banco de dados}: SQLite v3.8.10.2;
    \item \textbf{Ambiente de suporte}: Google Chrome, para \textit{debug} do aplicativo;
    \item \textbf{Simuladores}: 
    \begin{itemize}
        \item \textbf{iOS}: iPhone 6S Plus
        \item \textbf{Android}: Nexus 7
    \end{itemize}
    %\item \textbf{Tempo de desenvolvimento}: Ambos os integrantes trabalharam \textcolor{red}{XX} horas;
\end{itemize}

\subsection{Relato de desenvolvimento} \label{subsec:experienciasdev}

É apresentado, a seguir, o relato das experiências obtidas durante o desenvolvimento com Ionic, divido nas funcionalidades 
já descritas na Subseção~\ref{subsection:planejamentodesenvolvimento} deste trabalho.
Há, ainda, um tópico de observações gerais pertinentes à implementação do \textit{app} como um todo.


 \begin{itemize}
 	
 	\item \textbf{Listagem de Remédios e Alertas};
 	
 	\textcolor{red}{teve que criar tabs diferentes para ios e para android, pois a composicao original do aplicativo nao caia bem no android, entao foi necessario criar duas views para poder melhor adaptar ao android}
 	Ainda assim, o ionic conseguiu deixar com aparencia bem nativa, mt legal ele mudar para um componente que é semelhante ao padrao da plataforma especifica
 	
 	Filtragem em vencidos e válidos
 	
 	Criacao da lista, mt simples
 	
 	
 	\item \textbf{Cadastro de Remédios};
 	
 	\textcolor{red}{celulas colapsaveis na tela de add remedio}
 	
 	\textcolor{red}{passar dados de uma pagina para outra, semelhante ao protocolo do ios em questao de dificuldade} (tb usando no cadastro de alertas)
 	
 	Acesso a camera [tirar foto da selecao e comparar, foto doremedio ou receita]
 	
 	Acesso a library de fotos
 	
 	Action sheet, igual ao nativo no ios e fez algo correspondente no android
 	
 	Alerta, mts simples tb
 	
 	uso do banco de dados
 	
 	picker view [foto comparando do de intervalo]
 	
 	foto e comparar a selecao do local
 	
 	
 	
 	\item \textbf{Cadastro de Farmácias};
 	
 	\textcolor{red}{PARECER SOBRE A UTILIZAÇÃO DO GOOGLE MAPS - A utilização do mapa se deu por meio da \textit{API Maps} do Google. Não houveram grandes dificuldades na criação do mapa e renderização do mesmo na tela. Apenas por meio de pesquisas e estudos na documentação
 		da \textit{API} já foram suficientes para a elaboração do que precisava ser feito. }
 	
 	Ligação
 	
 	
 	\item \textbf{Cadastro de Alertas};
 	
 	\textcolor{red}{datepicker para selecionar data, foi utilizado um projeto disponivel no github para a exibição de um date picker para o cadastro de alertas, simples e fácil de se utilizar} (tb usado no cadastro de remedios) [foto comparando]
 	
 	time picker [foto comparando]
 	
 	Notificações
 	
 	
 	\item \textbf{Observações gerais}; 	
 	
 	\textcolor{red}{nao é possivel debugar o ios pelo terminal, por causa da atualizacao do ios 9, por isso é necessario do xcode ou do safari, ou seja um mac}
 	\textcolor{red}{usar o banco na primeira controladora do sistema, cordova nao carrega antes da view}
 	\textcolor{red}{possibilidade de debugar com os navegadores, é possivel usar os navegadores para poder debugar enquanto esta rodando no simulador das plataformas}
 	\textcolor{red}{debug com alerta de javascript, nao eh falado em lugar nenhum, mas facilita muito. foi encontrado por pesquisas na comunidade on line}
 	\textcolor{red}{ng-class nao funciona na ion-nav e foi muito dificil descobrir isso}
 	\textcolor{red}{problemas na hora de executar no iphone, o projeto não executa corretamente as vezes por motivo desconhecido}
 	
 \end{itemize} 
 
 \begin{figure}
 	\centering
 	\begin{minipage}{.5\textwidth}
 		\centering
 		\includegraphics[width=.8\linewidth]{ionic_lista_ios}
 	\end{minipage}%
 	\begin{minipage}{.5\textwidth}
 		\centering
 		\includegraphics[width=0.9\linewidth]{ionic_lista_android}
 	\end{minipage}
  \caption[Telas da lista de remédios (iOS \textit{versus} Android)]{ Telas da lista de remédios (iOS \textit{versus} Android). Fonte: Elaborado pelos autores.}
  \label{fig:ionic_lista}
 \end{figure}

 \begin{figure}
 	\centering
 	\begin{minipage}{.5\textwidth}
 		\centering
 		\includegraphics[width=.8\linewidth]{ionic_addremedio_ios}
 	\end{minipage}%
 	\begin{minipage}{.5\textwidth}
 		\centering
 		\includegraphics[width=0.9\linewidth]{ionic_addremedio_android}
 	\end{minipage}
 	\caption[Telas do cadastro de remédio (iOS \textit{versus} Android)]{ Telas do cadastro de remédio (iOS \textit{versus} Android). Fonte: Elaborado pelos autores.}
 	\label{fig:ionic_addremedio}
 \end{figure}

