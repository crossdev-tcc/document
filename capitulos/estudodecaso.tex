\chapter{Estudo de Caso} \label{estudodecaso}

Texto introdutório da seção

\section{Descrição do projeto selecionado} \label{subsec:descricaodoprojeto}

Mini Farma é um aplicativo criado inicialmente para a plataforma iOS que serve para controle dos medicamentos que as 
pessoas possuem em casa em suas ``farmacinhas'' particulares.\textcolor{red}{INSERIR PRINT DA TELA INICIAL?}


O aplicativo utiliza recursos nativos do sistema, os quais são descritos a seguir. 
\begin{itemize}
	\item \textbf{Localização geográfica}: Define a posição geográfica do dispositivo que o usuário está utilizando para poder permitir
    que seja salva a localização da farmácia onde ele comprou um medicamento para, caso seja necessário, explicar para alguém onde a 
    farmácia fica. Também é possível com base nessa localização da farmácia e do usuário, traçar uma rota para levá-lo diretamente à 
    farmácia em questão;
	\item \textbf{Notificações locais}: Diferente das notificações \textit{Push} as notificações locais são criadas e agendadas na central 
    de notificações do dispositivo e o sistema se encarrega de entregá-las corretamente de acordo com parâmentros definidos pelo aplicativo. 
    No caso do Mini Farma, as notificações são usadas para lembrar o usuário, na data e hora corretas, de que o mesmo deve tomar seus medicamentos;
	\item \textbf{Ligação}: As notificações podem conter ações que executam um determinado bloco de código dentro do aplicativo. No Mini Farma, 
    uma das notificações possíveis é a de aviso de pouca quantidade ou remédio esgotado, no caso do medicamento ter acabado. 
    Quando essa notificação é enviada, pode ser feita uma ligação diretamente pela ação da notificação para o número da 
    farmácia cadastrado no aplicativo. Dessa forma, o usuário pode solicitar uma nova quantidade de medicamento diretamente com a farmácia. 
	\item \textbf{Câmera}: Para facilitar a identificação dos medicamentos, é possível tirar uma foto com a câmera do dispositivo para 
    cada remédio cadastrado. Além de uma foto para o medicamento em si, é possível tirar uma foto da receita dele, caso haja;
    \item \textbf{Rolo de câmera}: Se o usuário já tiver uma foto que o ajude a identificar o seu medicamento ou da receita do mesmo salva 
    no rolo de câmera, é possível escolher a foto sem precisar tirar uma nova;
\end{itemize}

\textcolor{red}{\textit{* Vale ressaltar, que com excessão da Ligação, todos os outros recursos nativos do sistema, precisam necessariamente serem autorizados pelo
usuário para funcionar. Caso o usuário não autorize, o aplicativo ficará com funcionalidades reduzidas.}} 


Como banco de dados foi usado o SQLite, através do \textit{framework} externo e \textit{open-source FMDB}, 
disponível no \href{https://github.com/ccgus/fmdb}{Github};

A arquitetura do sistema foi criada com base no padrão de projeto MVC, possuindo Objetos de Acesso a Dados (DAO) para comunicação 
com o banco de dados. \textcolor{red}{No entanto, o MVC tradicional difere em alguns aspectos do MVC usado no \textit{iOS}.} 

\textcolor{red}{No \textit{iOS} o MVC trás uma controladora ligada à classe de \textit{view}, o que é chamado de \textit{ViewController} que é responsável por 
criar a ponte entre a interação do usuário com as classes modelos e a \textit{view}. Dessa forma, as \textit{ViewControllers} têm a 
responsabilidade de repassar as entradas do usuário para a modelo e controlar a \textit{view} para apresentar os resultados que modelo retornar.}


\textcolor{red}{Diferente do MVC tradicional no qual a controladora apenas repassa as informações que estão entrando na fronteira da aplicação, 
as \textit{ViewControllers} têm, além dessa responsabilidade, também a responsabilidade de instanciar e gerenciar a \textit{view} no tocante
a organização dos elementos e apresentação das informações.}


\textcolor{red}{Outra observação sobre as diferenças entre os dois MVC's, é que no MVC tradicional pode haver uma controladora apenas 
para todas as modelos ou uma controladora por modelo, mas nenhuma ligada diretamente a uma \textit{view}, no entanto, no iOS há uma 
\textit{ViewController} por \textit{view}, podendo haver ainda uma controladora relacionada com cada modelo conforme o padrão tradicional do MVC.}

\textcolor{yellow}{\textit{TALVEZ SERIA BOM COLOCAR UMAS IMAGENS AQUI PARA MOSTRAR AS DIFERENCAS ENTRE OS DOIS MVCs GRAFICAMENTE, V-C-M E V-VC-C-M-DAO}}

\section{Ambiente de desenvolvimento} \label{subsec:ambientedesenvolvimento}

O projeto do aplicativo Mini Farma foi executado contando com uma equipe de dois integrantes com 
conhecimento intermediário na plataforma \textit{iOS}. Os detalhes técnicos são listados a seguir.
\begin{itemize}
    \item \textbf{Máquinas}: Dois MacBook Pro Retina 13", processador \textit{Intel Core i5}, 8 GB de \textit{RAM};
    \item \textbf{Sistema Operacional}: \textit{Mac OS X Yosemite};
    \item \textbf{IDE}: \textit{Xcode} v6.4;
    \item \textbf{Linguagem de Programação}: \textit{Swift} v1.3;
    \item \textbf{Tempo de desenvolvimento}: Ambos os integrantes trabalharam no projeto quatro horas por dia de segunda à sexta por duas semanas;
\end{itemize}

A partir do aplicativo feito em \textit{iOS}, foi construída uma versão utilizando o \textit{Ionic} para as plataformas \textit{Android} e \textit{iOS}.
Para a criação dessa versão do \textit{app}, a equipe foi alterada, mas permaneceu com dois integrantes, que são os autores deste trabalho.
No entanto, os integrantes não possuiam qualquer conhecimento prévio no \textit{framework Ionic}
e apenas possuiam um conhecimento básico em \textit{HTML}, \textit{CSS} e \textit{JavaScript}. Os detalhes técnicos são listados a seguir.
   
\begin{itemize}
    \item \textbf{Máquinas}: Dois MacBook Pro Retina 13", processador \textit{Intel Core i5}, 8 GB de \textit{RAM};
    \item \textbf{Sistema Operacional}: \textit{Mac OS X El Capitan};
    \item \textbf{IDE}: \textit{WebStorm} v2016.1.1;
    \item \textbf{Linguagem de Programação}: \textit{JavaScript}. 
    \begin{itemize}
        \item \textit{HTML} e \textit{CSS} foram usados paralelamente para a criação da interface gráfica do aplicativo;
    \end{itemize}
    \item \textbf{Ambiente de suporte}: \textit{Google Chrome}, para \textit{debug} do aplicativo;
    \item \textbf{Tempo de desenvolvimento}: Ambos os integrantes trabalharam x horas;
\end{itemize}


\begin{comment}
será que aqui fala das versoes do ios, ionic, cordova, angularjs? sim e explicar pq nao escolheu a versao 2 do ionic, pq eh recente e estavel
\end{comment}


\section{Desenvolvimento multiplataformas do projeto} \label{sec:desenvolvimentomulti}

Desenvolvimento multi se deu da seguinte forma......