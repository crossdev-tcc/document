\chapter{Desenvolvimento de aplicativos móveis}\label{referencialteorico}

Aqui serão descritos as arquiteturas móveis mais conhecidas e utilizadas no mercado. São elas \textit{iOS} e \textit{Android}.

\section{iOS}

O sistema operacional \textit{iOS} foi lançado juntamente com o primeiro \textit{iPhone} criado em Janeiro de 2007 e 
funciona como uma interface entre as aplicações desenvolvidas pelos programadores (\textit{apps}) e o \textit{hardware} 
dos dispositivos (\textit{iPhone}, \textit{iPad}, \textit{iPod}). Desta forma a comunicação com o \textit{hardware} do dispositivo se dá 
por meio de um conjunto bem definido de interfaces do sistema, o que facilita o desenvolvimento de apps 
que funcionam corretamente entre os variados tipos de hardware dos dispositivos da \textit{Apple}. Sua arquitetura 
é baseada em camadas, conforme a figura a seguir.
 
%\begin{figure}
%  \includegraphics[width=\linewidth]{SystemLayers_2x.png}
%  \caption{Arquitetura em camadas do iOS.}
%  \label{fig:iosLayers}
%\end{figure}

%A Figura \ref{fig:iosLayers} Mostrar as camadas da plataforma \textit{iOS}.
\begin{itemize}

	\item Cocoa Touch Layer
	\item Media Layer;
	\item Core Services Layer;
	\item Core OS Layer;

\end{itemize}

Como recomendação, a Apple explica que deve-se preferir o uso de camadas mais elevadas, pois as camadas de 
mais alto nível possuem abstrações orientadas a objeto de funcionalidades das camadas mais baixas. Isso 
torna o desenvolvimento mais fácil, pois reduz a quantidade de código que deve ser criado e mantém funcionalidade 
complexas das camadas mais baixas encapsuladas por meio das interfaces. No entanto, não há problema em usar 
funcionalidades presentes nas camadas mais baixas, se essas não estiverem disponíveis por meio de abstrações nas 
camadas mais elevadas.A maioria das interfaces disponíveis para uso são disponibilizadas por meio de frameworks, 
que são diretórios, que podem ser adicionados ao projeto no Xcode, contendo DSL's e os recursos necessários como, 
imagens, apps auxiliares e arquivos header, para a biblioteca funcionar corretamente.

\begin{itemize}

	\item A camada Cocoa Touch é a camada mais alto nível onde são fornecidos serviços básicos de interação 
    com o usuário como entrada baseada em toques e notificações Push e outras tecnologias necessárias para
     melhorar a experiencia do usuário como multitarefas, Continuidade (Handoff) e AirDrop, além de frameworks 
     de alto nível que permitem acesso a funcionalidades do sistema como AddressBook para contatos, EventKit 
     para eventos relacionados ao calendário e MapKit para mapas.
	\item A camada logo abaixo da Cocoa Touch é a camada Media que contém tecnologias e frameworks necessários 
    para a implementação de experiencias multimedia com áudio, vídeo e gráficos.
	\item A próxima camada, logo abaixo da camada Media, é a camada Core Services. Essa camada está mais próxima 
    do hardware e portanto possui acesso a funcionalidades de mais baixo nível como localização, telefonia, threads 
    e SQLite. Aqui residem dois dos frameworks mais importantes do iOS que são o Foundation e o Core Foundation, 
    ambos relacionados com o gerenciamento de dados e alguns serviços e definem todos os tipos básicos de dados que 
    todos os apps usam, como por exemplo, coleções, strings, data e hora, sockets e threads.
	\item A última camada é a camada Core OS, na qual as funcionalidades de mais baixo nível são construidas e 
    provavelmente utilizadas por outros frameworks em outras camadas. Se a aplicação possui requisitos de segurança 
    ou comunicação com acessórios externos mais complicados, é possível usar as funcionalidades dessa camada.

\end{itemize}

\section{Android}

Arquitetura

\begin{itemize}

	\item Arquitetura 1
    \item Arquitetura 2
    \item Arquitetura 3 

\end{itemize}

\section{Ionic}

\textit{Ionic} é um \textit{framework} gratuito e \textit{open source} para desenvolvimento de aplicativos híbridos utilizando tecnologias 
\textit{web} como \textit{HTML}, \textit{CSS} e \textit{JavaScript} otimizadas para dispositivos móveis. Com apenas um código é possível criar aplicativos para diversas plataformas como
\textit{iOS} e \textit{Android}. %Criado sob o \textit{framework} Cordova que permite acesso a funcionalidades nativas dos dispositivos.

Cordova é um \textit{framework} para desenvolvimento de aplicativos híbridos criado pela empresa Nitobi. 
Após ser comprada pela Adobe Systems Inc. teve seu código doado para a Apache para garantir que outras empresas pudessem contribuir, já que muitas empresas confiavam e conheciam as licenças da Apache.
Como o PhoneGap é uma marca registrada de propriedade da Adobe, na Apache teve seu nome alterado para Cordova.
Tanto o Apache Cordova quanto o Adobe PhoneGap, são gratuitos e \textit{open sources}, no entanto, o PhoneGap possui um ambiente integrado com serviços da Adobe como o PhoneGap Build.

%e seu código liberado em uma versão \textit{open source}, onde grandes empresas, como Google, Mozilla e Microsoft, contribuiram para o projeto. 
Cordova permite que aplicações não nativas tenham acesso a funcionalidades nativas do dispositivo como sensores, 
contatos e camera. No entanto, o Cordova apenas consegue fazer um aplicativo criado em 
\textit{HTML} rodar como se fosse nativo em um dispositivo com \textit{Android} e \textit{iOS}, mas não consegue 
imitar a usabilidade e aparência dos dispositivos nativos. 
Para preencher essa lacuna, outros \textit{frameworks} foram criados em cima do Cordova como um complemento, provendo bibliotecas \textit{HTML} e \textit{CSS} para a 
criação de um \textit{front-end} que se aproxime o máximo possível da usabilidade e experiência de usuário de aplicações nativas.

O Cordova possui muitos plugins para poder acessar funcionalidades específicas do aparelho em que está sendo executado. Esses devem ser instalados no projeto do aplicativo via \textit{Command-Line Interface} (CLI). 

A fim de se evitar confusões em relação aos nomes dados ao PhoneGap e ao Cordova, pode-se entender o PhoneGap como uma distribuição criada pela Adobe do Apache Cordova. 


\begin{itemize}

    \item Cordova
	
    \item Ionic
    
    \item AngularJS

\end{itemize}