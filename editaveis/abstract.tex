\begin{resumo}[Abstract]
 \begin{otherlanguage*}{english}
  
  \begin{comment}
  O desenvolvimento de aplicativos móveis intensificou-se nos últimos anos devido ao crescimento acelerado e popularização dos smartphones. 
  Cada empresa do ramo móvel possui seu próprio sistema operacional, loja de aplicativos, parcela de mercado e ambientes de desenvolvimento. 
  No entanto, quanto mais sistemas diferentes existem, maior o esforço, custo e tempo para 
  desenvolver um app para todas as plataformas existentes. 
  Visando sanar dificuldades, o desenvolvimento multiplataforma tem a premissa de criação de apenas um código que possa abranger várias plataformas. 
 Nesse contexto, o presente trabalho busca definir vantagens e desvantagens da abordagem multiplataforma quando comparada com a abordagem nativa. 
 Por meio da análise de um exemplo de uso, que consistiu na recriação de um aplicativo 
 nativo, implementado originalmente para a plataforma iOS, utilizando o framework Ionic, foi possível comparar e comprovar empiricamente os dados obtidos na literatura. 
 Mediante análise exploratória, foram selecionadas funcionalidades para verificar a viabilidade do desenvolvimento dessas
 em ambiente multiplataforma e para realizar uma comparação com o desenvolvimento das mesmas em um ambiente nativo. Os dados 
 obtidos foram confrontados com as opiniões de especialistas para avaliar a percepção dos mesmos sobre o cenário atual do 
 desenvolvimento multiplataforma. Concluiu-se que cada abordagem tem um momento certo 
 para ser utilizada, não sendo uma melhor que a outra, mas apenas diferentes entre si, e deve-se avaliar cada caso, 
 considerando-se uma série de fatores para escolher qual abordagem utilizar.
  \end{comment}

  Mobile application development has intensified in recent years due to rapid growth and popularization of smartphones. Each company of 
  mobile branch has its own operating system, application store, market share and development environments. 
  However, the more different systems exist, the greater the effort, cost and time to develop an app for all existing platforms. Then came the concept of cross-platform development, 
  with the premise of code only once and span multiple platforms. In this context, this paper seeks to define advantages and disadvantages of the cross-platform approach 
  compared with the native approach. Through an analysis of study, which consisted of the recreation of a native application, originally implemented for iOS platform, 
  using the Ionic framework, it was possible to compare and empirically verify the data obtained in the literature. Through exploratory analysis, a small group of key features were selected
  in order to verify their development viability on cross-platform enviroment and then compare with their native development. The data obtained were compared with experts opinions to 
  analize their perception of cross-platform development current scenario. It was concluded at the end of this work, that each approach has a certain time 
  to be used, not being better than the other, but only different, and developers must evaluate each case considering a number of factors to choose which approach is better to be used in each context.

  \vspace{\onelineskip}

  \noindent 
  \textbf{Key-words}: development. mobile. cross. platform. native. ionic.
 \end{otherlanguage*}
\end{resumo}
