\begin{resumo}[Abstract]
 \begin{otherlanguage*}{english}
  
  \begin{comment}
  O desenvolvimento de aplicativos móveis intensificou-se nos últimos anos devido ao crescimento acelerado e popularização dos \textit{smartphones}. Cada empresa
  do ramo móvel possui seu próprio sistema operacional, loja de aplicativos, parcela de mercado e ambientes de desenvolvimento. No entanto, quanto mais 
  sistemas diferentes existem, maior o esforço, custo e tempo para desenvolver um \textit{app} para todas as plataformas existentes. Surgiu então, o conceito de desenvolvimento multiplataformas,
  com a premissa de codificar apenas uma vez e abranger várias plataformas. Neste contexto, o presente trabalho busca definir vantagens e desvantagens da abordagem
  multiplataformas quando comparada com a abordagem nativa. Por meio de um estudo de caso, que consistiu na recriação de um aplicativo nativo, implementado originalmente para a plataforma iOS,
  utilizando o \textit{framework} Ionic, foi possível comparar e comprovar empiricamente os dados obtidos na literatura. As ferramentas multiplataformas, atualmente,
  não possuem mais as limitações apontadas pela literatura, pois evoluíram muito rapidamente ao longo dos últimos cinco anos. Concluiu-se, ao fim desse trabalho, que cada abordagem tem um momento certo
  para ser utilizada, não sendo uma melhor que a outra, mas apenas diferentes entre si, e deve-se avaliar cada caso, considerando-se uma série de fatores para escolher qual abordagem utilizar.
  \end{comment}

  Mobile application development has intensified in recent years due to rapid growth and popularization of smartphones. Each company of 
  mobile branch has its own operating system, application store, market share and development environments. 
  However, the more different systems exist, the greater the effort, cost and time to develop an app for all existing platforms. Then came the concept of cross-platform development, 
  with the premise of code only once and span multiple platforms. In this context, this paper seeks to define advantages and disadvantages of the cross-platform approach 
  compared with the native approach. Through a case study, which consisted of the recreation of a native application, originally implemented for iOS platform, 
  using the Ionic framework, it was possible to compare and empirically verify the data obtained in the literature. Cross-platform tools currently 
  no longer have the limitations mentioned in the literature, they evolved very rapidly over the last five years. It was concluded at the end of this work, that each approach has a certain time 
  to be used, not being better than the other, but only different, and developers must evaluate each case considering a number of factors to choose which approach is better to be used in each context.

  \vspace{\onelineskip}

  \noindent 
  \textbf{Key-words}: development. mobile. cross. platform. native. ionic.
 \end{otherlanguage*}
\end{resumo}
