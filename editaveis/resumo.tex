\begin{resumo}
\begin{comment}
 O resumo deve ressaltar o objetivo, o método, os resultados e as conclusões 
 do documento. A ordem e a extensão
 destes itens dependem do tipo de resumo (informativo ou indicativo) e do
 tratamento que cada item recebe no documento original. O resumo deve ser
 precedido da referência do documento, com exceção do resumo inserido no
 próprio documento. (\ldots) As palavras-chave devem figurar logo abaixo do
 resumo, antecedidas da expressão Palavras-chave:, separadas entre si por
 ponto e finalizadas também por ponto. O texto pode conter no mínimo 150 e 
 no máximo 500 palavras, é aconselhável que sejam utilizadas 200 palavras. 
 E não se separa o texto do resumo em parágrafos.

 O desenvolvimento de aplicativos móveis cresceu muito nos últimos anos
 devido ao crescimento acelerado dos \textit{smartphones}. Com isso, surgiram 
 lojas de aplicativos, ambientes e \textit{SDKs} diferentes para 
 cada sistema operacional móvel. No entanto, quanto mais sistemas diferentes 
 existem, maior o esforço para desenvolver os \textit{apps} para todas 
 as plataformas. Com essa gama de plataformas diferentes, invariavelmente,
 o desenvolvedor precisará escolher qual plataforma irá seguir. Surge, então, uma forma 
 de desenvolvimento com a premissa de resolver 
 o problema da criação de um \textit{apps} para cada plataforma. 
 Essa solução ficou conhecida como desenvolvimento multiplataformas e possui
 vantagens e desvantagens em relação ao desenvolvimento nativo. Será feita uma
 pesquisa e um estudo de caso para avaliar essas vantagens e desvantagens de 
 maneira empírica e assim definir uma forma de escolher qual abordagem deve ser 
 adotada em cada caso específico.
\end{comment}

 O desenvolvimento de aplicativos móveis intensificou-se nos últimos anos devido ao crescimento acelerado dos \textit{smartphones}. Cada empresa
 do ramo móvel possui seu próprio sistema operacional, loja de aplicativos, parcela de mercado e ambientes de desenvolvimento. No entanto, quanto mais 
 sistemas diferentes existem, maior o esforço, custo e tempo para desenvolver os \textit{apps} para todas as plataformas. Surgiu então, o conceito de desenvolvimento multiplataformas,
 com a premissa de codificar apenas uma vez e abranger várias plataformas. Diante desse panorama, o presente trabalho busca definir vantagens e desvantagens da abordagem
 multiplataformas quando comparada com a abordagem nativa. Por meio de um estudo de caso, que consistiu na recriação de um aplicativo nativo, implementado originalmente para a plataforma iOS,
 utilizando o \textit{framework} Ionic, foi possível comparar e comprovar empiricamente os dados obtidos na literatura. As ferramentas multiplataformas, atualmente,
 não possuem mais as limitações apontadas pela literatura, pois evoluíram muito rapidamente ao longo dos últimos cinco anos. Concluiu-se, ao fim desse trabalho, que cada abordagem tem um momento certo
 para ser utilizada, não sendo uma melhor que a outra, mas apenas diferentes entre si, e deve-se avaliar cada caso, considerando-se uma série de fatores para escolher qual abordagem utilizar.

 \vspace{\onelineskip}
    
 \noindent
 \textbf{Palavras-chaves}: desenvolvimento. móvel. cross. nativo. ionic. 
\end{resumo}
