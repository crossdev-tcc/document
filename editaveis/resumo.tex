\begin{resumo}

 O desenvolvimento de aplicativos móveis intensificou-se nos últimos anos devido ao crescimento acelerado e popularização dos \textit{smartphones}. Cada empresa
 do ramo móvel possui seu próprio sistema operacional, loja de aplicativos, parcela de mercado e ambientes de desenvolvimento. No entanto, quanto mais 
 sistemas diferentes existem, maior o esforço, custo e tempo para desenvolver um \textit{app} para todas as plataformas existentes. Surgiu então, o conceito de desenvolvimento multiplataforma,
 com a premissa de codificar apenas uma vez e abranger várias plataformas. Nesse contexto, o presente trabalho busca definir vantagens e desvantagens da abordagem
 multiplataforma quando comparada com a abordagem nativa. Por meio da análise de um exemplo de uso, que consistiu na recriação de um aplicativo nativo, implementado originalmente para a plataforma iOS,
 utilizando o \textit{framework} Ionic, foi possível comparar e comprovar empiricamente os dados obtidos na literatura. As ferramentas multiplataforma, atualmente,
 não possuem mais as limitações apontadas pela literatura, pois evoluíram muito rapidamente ao longo dos últimos cinco anos. Concluiu-se, ao fim deste trabalho, que cada abordagem tem um momento certo
 para ser utilizada, não sendo uma melhor que a outra, mas apenas diferentes entre si, e deve-se avaliar cada caso, considerando-se uma série de fatores para escolher qual abordagem utilizar.

 \vspace{\onelineskip}
    
 \noindent
 \textbf{Palavras-chaves}: desenvolvimento. móvel. multiplataforma. nativo. ionic. 
\end{resumo}
