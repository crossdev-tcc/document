\begin{resumo}
\begin{comment}
O resumo deve ressaltar o objetivo, o método, os resultados e as conclusões 
 do documento. A ordem e a extensão
 destes itens dependem do tipo de resumo (informativo ou indicativo) e do
 tratamento que cada item recebe no documento original. O resumo deve ser
 precedido da referência do documento, com exceção do resumo inserido no
 próprio documento. (\ldots) As palavras-chave devem figurar logo abaixo do
 resumo, antecedidas da expressão Palavras-chave:, separadas entre si por
 ponto e finalizadas também por ponto. O texto pode conter no mínimo 150 e 
 no máximo 500 palavras, é aconselhável que sejam utilizadas 200 palavras. 
 E não se separa o texto do resumo em parágrafos.
\end{comment}

 O desenvolvimento de aplicativos móveis cresceu muito nos últimos anos
 devido ao crescimento acelerado dos \textit{smartphones}, com suas 
 lojas de aplicativos, ambientes e \textit{SDK's} diferentes para 
 cada sistemas operacional. No entanto, quanto mais sistemas diferentes 
 existem, maior o esforço para desenvolver os \textit{apps} para todas 
 as plataformas. Com essa gama de plataformas diferentes existentes 
 atualmente, surge uma forma de desenvolvimento com a premissa de resolver 
 o problema da criação de vários \textit{apps} iguais para várias plataformas. 
 Essa solução ficou conhecida como desenvolvimento multiplataformas e possui
 vantagens e desvantagens em relação ao desenvolvimento nativo. Será feita uma
 pesquisa e um estudo de caso para avaliar essas vantagens e desvantagens de 
 maneira empírica e assim definir uma forma de escolher qual abordagem deve ser 
 adotada em cada caso específico. 

 \vspace{\onelineskip}
    
 \noindent
 \textbf{Palavras-chaves}: desenvolvimento. móvel. cross. nativo. 
\end{resumo}
