\part{Meus Referenciais Teóricos}

\chapter[Arquiteturas móveis]{Arquiteturas Móveis}


Aqui serão descritos as arquiteturas móveis mais conhecidas e utilizadas no mercado. São elas iOS e Android.


\section{iOS}

O sistema operacional iOS foi lançado juntamente com o primeiro iPhone criado em Janeiro de 2007 e 
funciona como uma interface entre as aplicações desenvolvidas pelos programadores (apps) e o hardware 
dos dispositivos (iPhone, iPad, iPod). Desta forma a comunicação com o hardware do dispositivo se dá 
por meio de um conjunto bem definido de interfaces do sistema, o que facilita o desenvolvimento de apps 
que funcionam corretamente entre os variados tipos de hardware dos dispositivos da Apple. Sua arquitetura 
é baseada em camadas, conforme a figura X a seguir.

 
\begin{itemize}

	\item Cocoa Touch Layer
	\item Media Layer;
	\item Core Services Layer;
	\item Core OS Layer;

\end{itemize}


Como recomendação, a Apple explica que deve-se preferir o uso de camadas mais elevadas, pois as camadas de 
mais alto nível possuem abstrações orientadas a objeto de funcionalidades das camadas mais baixas. Isso 
torna o desenvolvimento mais fácil, pois reduz a quantidade de código que deve ser criado e mantém funcionalidade 
complexas das camadas mais baixas encapsuladas por meio das interfaces. No entanto, não há problema em usar 
funcionalidades presentes nas camadas mais baixas, se essas não estiverem disponíveis por meio de abstrações nas 
camadas mais elevadas.A maioria das interfaces disponíveis para uso são disponibilizadas por meio de frameworks, 
que são diretórios, que podem ser adicionados ao projeto no Xcode, contendo DSL's e os recursos necessários como, 
imagens, apps auxiliares e arquivos header, para a biblioteca funcionar corretamente.


\begin{itemize}

	\item A camada Cocoa Touch é a camada mais alto nível onde são fornecidos serviços básicos de interação 
    com o usuário como entrada baseada em toques e notificações Push e outras tecnologias necessárias para
     melhorar a experiencia do usuário como multitarefas, Continuidade (Handoff) e AirDrop, além de frameworks 
     de alto nível que permitem acesso a funcionalidades do sistema como AddressBook para contatos, EventKit 
     para eventos relacionados ao calendário e MapKit para mapas.
	\item A camada logo abaixo da Cocoa Touch é a camada Media que contém tecnologias e frameworks necessários 
    para a implementação de experiencias multimedia com áudio, vídeo e gráficos.
	\item A próxima camada, logo abaixo da camada Media, é a camada Core Services. Essa camada está mais próxima 
    do hardware e portanto possui acesso a funcionalidades de mais baixo nível como localização, telefonia, threads 
    e SQLite. Aqui residem dois dos frameworks mais importantes do iOS que são o Foundation e o Core Foundation, 
    ambos relacionados com o gerenciamento de dados e alguns serviços e definem todos os tipos básicos de dados que 
    todos os apps usam, como por exemplo, coleções, strings, data e hora, sockets e threads.
	\item A última camada é a camada Core OS, na qual as funcionalidades de mais baixo nível são construidas e 
    provavelmente utilizadas por outros frameworks em outras camadas. Se a aplicação possui requisitos de segurança 
    ou comunicação com acessórios externos mais complicados, é possível usar as funcionalidades dessa camada.

\end{itemize}



\section{Desenvolvimento}

O Desenvolvimento (Miolo ou Corpo do Trabalho) é subdividido em seções de 
acordo com o planejamento do autor. As seções primárias são aquelas que 
resultam da primeira divisão do texto do documento, geralmente 
correspondendo a divisão em capítulos. Seções secundárias, terciárias, 
etc., são aquelas que resultam da divisão do texto de uma seção primária, 
secundária, terciária, etc., respectivamente.

As seções primárias são numeradas consecutivamente, seguindo a série 
natural de números inteiros, a partir de 1, pela ordem de sua sucessão no 
documento.

O Desenvolvimento é a seção mais importante do trabalho, por isso exigi-se 
organização, objetividade e clareza. É conveniente dividi-lo em pelo menos 
três partes:

\begin{itemize}

	\item Referencial teórico, que corresponde a uma análise dos trabalhos 
	relevantes, encontrados na pesquisa bibliográfica sobre o assunto. 
	\item Metodologia, que é a descrição de todos os passos metodológicos 
	utilizados no trabalho. Sugere-se que se enfatize especialmente em (1) 
	População ou Sujeitos da pesquisa, (2) Materiais e equipamentos 
	utilizados e (3) Procedimentos de coleta de dados.
	\item Resultados, Discussão dos resultados e Conclusões, que é onde se 
	apresenta os dados encontrados a análise feita pelo autor à luz do 
	Referencial teórico e as Conclusões.

\end{itemize}

\section{Uso de editores de texto}

O uso de programas de edição eletrônica de textos é de livre escolha do autor. 

